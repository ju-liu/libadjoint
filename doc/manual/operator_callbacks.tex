\chapter{Operator callbacks}
\minitoc
\vspace{\fill}
\newpage

\section{Operator callbacks}
When the developer annotates a model, he must give a label to each
block that appears in the annotation, e.g.\ \mint{c}|"MassMatrix"|, \mint{c}|"Identity"|,
\mint{c}|"AdvectionOperator"|, etc. \libadjoint is capable of the formal manipulation
of this annotation to say how the adjoint equation should be assembled;
but it is not possible for \libadjoint to actually assemble the adjoint
equations while these names are merely abstract labels.

However, if the model developer associates each name appearing in the annotation
with a function that allows \libadjoint to use that block, then \libadjoint
can actually assemble each adjoint equation. 

This suggests a clean division
of labour: the model developer describes the model in an annotation, and supplies
callbacks for each individual component of that annotation; \libadjoint manipulates
the annotation to decide how the individual components should all be put together
to assemble each adjoint equation. This division of labour turns out to be
an excellent way to write an adjoint model: the complex manipulation of the annotation
can be written in \libadjoint precisely once, and the model developer can automatically
test the correctness of the annotation and operator callbacks (\autoref{chap:debugging}).

\defapis{adj_register_operator_callback}
\begin{framed}
\begin{minipage}{\columnwidth}
\begin{ccode}
int adj_register_operator_callback(adj_adjointer* adjointer, int type,
                                   char* name, void (*fn)(void));
\end{ccode}
\begin{fortrancode}
  function adj_register_operator_callback(adjointer, type, name, fnptr) 
           result(ierr)
    type(adj_adjointer), intent(inout) :: adjointer
    integer(kind=c_int), intent(in) :: type
    character(len=*), intent(in) :: name
    type(c_funptr), intent(in), value :: fnptr
    integer(kind=c_int) :: ierr
  end function adj_register_operator_callback
\end{fortrancode}
\end{minipage}
\end{framed}
This function registers a given operator callback with blocks of name \texttt{name}
so that \libadjoint can use it when necessary. For a list of the valid values of
\texttt{type}, see \autoref{sec:operator_callback_types}. If an operator callback
has not been registered, and \libadjoint needs it to perform some task that the model
developer has requested, the function will return an \refapi{ADJ_ERR_NEED_CALLBACK}
error.

Example usage in C:
\begin{framed}
\begin{minipage}{\columnwidth}
\begin{ccode}
  ierr = adj_register_operator_callback(adjointer, ADJ_NBLOCK_ACTION_CB, 
                                   "AdvectionOperator",
                                   (void (*)(void)) advection_action_proc);
  adj_chkierr(ierr);
\end{ccode}
\end{minipage}
\end{framed}

Example usage in Fortran:
\begin{framed}
\begin{minipage}{\columnwidth}
\begin{fortrancode}
  ierr = adj_register_operator_callback(adjointer, ADJ_BLOCK_ASSEMBLY_CB, 
                                    "MassMatrix",
                                    c_funloc(mass_matrix_assembly_proc));
  call adj_chkierr(ierr);
\end{fortrancode}
\end{minipage}
\end{framed}

As \libadjoint is only given the function pointer to the operator callback, it cannot check that the
interface of the given function is correct. If you register a function which is declared to take 
in an interface different to that which \libadjoint expects, then the model will segfault when
\libadjoint calls that function (or worse, smash the stack). Be very careful when writing operator callbacks to use exactly
the interface documented in this manual!
\subsection{Operator callback types} \label{sec:operator_callback_types}
\defapiss{ADJ_BLOCK_ACTION_CB}
\begin{framed}
\begin{minipage}{\columnwidth}
\begin{ccode}
  void block_action(int nvar, adj_variable* variables, adj_vector* dependencies,
                    int hermitian, adj_scalar coefficient, adj_vector input,
                    void* context, adj_vector* output);
\end{ccode}
\begin{fortrancode}
  subroutine block_action(nvar, variables, dependencies, hermitian, coefficient,
                          input, context, output) bind(c)
    integer(kind=c_int), intent(in), value :: nvar
    type(adj_variable), dimension(nvar), intent(in) :: variables
    type(adj_vector), dimension(nvar), intent(in) :: dependencies
    integer(kind=c_int), intent(in), value :: hermitian
    adj_scalar_f, intent(in), value :: coefficient
    type(adj_vector), intent(in), value :: input
    type(c_ptr), intent(in), value :: context
    type(adj_vector), intent(out) :: output
  end subroutine block_action
\end{fortrancode}
\end{minipage}
\end{framed}
This callback computes the action of a block on a specified input vector \texttt{input}.

\begin{framed}
%\begin{minipage}
Mathematical description: let $V$ represent the block with which this routine is associated. Then,
if \texttt{hermitian == ADJ_FALSE}, this routine must compute
\begin{equation*}
\texttt{output} = \texttt{coefficient} \cdot V(d_1,d_2,\dots,d_{\texttt{nvar}}) \cdot \texttt{input},
\end{equation*}
and if \texttt{hermitian == ADJ_TRUE}, this routine must compute
\begin{equation*}
\texttt{output} = \texttt{coefficient} \cdot V(d_1,d_2,\dots,d_{\texttt{nvar}})^* \cdot \texttt{input},
\end{equation*}
where $(d_1, d_2, \dots, d_{\texttt{nvar}})$ are the nonlinear dependencies of this block supplied in the \texttt{dependencies} array.
%\end{minipage}
\end{framed}

\begin{framed}
Detailed argument list:
\begin{description}
\item[\texttt{nvar}] The number of variables that this block depends on.
\item[\texttt{variables}] An array of \refapi{adj_variable}s indicating which dependencies are being supplied to this routine. The length of the
array is given by \texttt{nvar}. The first entry of the \texttt{variables} array
tells the model developer what the first entry of the \texttt{dependencies} is, the second pairs with the second, and so on.
\item[\texttt{dependencies}] An array of \refapi{adj_vector}s containing the values of the dependencies of this block. The dependencies supplied
here are the same as those which were registered during the annotation.
\item[\texttt{hermitian}] A flag indicating whether the routine is to compute the action of the block, or the action of the Hermitian of the
block. It will either be \texttt{ADJ_TRUE} or \texttt{ADJ_FALSE}.
\item[\texttt{coefficient}] A coefficient by which the routine must scale the output.
\item[\texttt{input}] The \refapi{adj_vector} to which the operator should be applied.
\item[\texttt{context}] This is a pointer to arbitrary model data, supplied at the time that the block was created during the annotation.
\item[\texttt{output}] This is an \refapi{adj_vector} which is to contain the output of the matrix-vector action.
\end{description}
\end{framed}

When necessary: this callback is necessary for all blocks that appear off the diagonal of the annotation; i.e., the block
targets an \refapi{adj_variable} in an \refapi{adj_equation} that is not the variable being solved for in that
\refapi{adj_equation}.

\defapiss{ADJ_BLOCK_ASSEMBLY_CB}
\defapiss{ADJ_NBLOCK_DERIVATIVE_ACTION_CB}
\defapiss{ADJ_NBLOCK_ACTION_CB}

\section{Source-term callbacks}
\defapis{adj_register_forward_source_callback}
\defapis{adj_register_functional_derivative_callback}

\section{Functional evaluation callbacks}
\defapis{adj_register_functional_callback}
