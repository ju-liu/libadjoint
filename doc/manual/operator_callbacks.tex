\chapter{Operator callbacks}
\minitoc
\vspace{\fill}
\newpage

\section{Operator callbacks}
When the developer annotates a model, he must give a label to each
block that appears in the annotation, e.g.\ \mint{c}|"MassMatrix"|, \mint{c}|"Identity"|,
\mint{c}|"AdvectionOperator"|, etc. \libadjoint is capable of the formal manipulation
of this annotation to say how the adjoint equation should be assembled;
but it is not possible for \libadjoint to actually assemble the adjoint
equations while these names are merely abstract labels.

However, if the model developer associates each name appearing in the annotation
with a function that allows \libadjoint to use that block, then \libadjoint
can actually assemble each adjoint equation. 

This suggests a clean division
of labour: the model developer describes the model in an annotation, and supplies
callbacks for each individual component of that annotation; \libadjoint manipulates
the annotation to decide how the individual components should all be put together
to assemble each adjoint equation. This division of labour turns out to be
an excellent way to write an adjoint model: the complex manipulation of the annotation
can be written in \libadjoint precisely once, and the model developer can automatically
test the correctness of the annotation and operator callbacks (\autoref{chap:debugging}).

\defapis{adj_register_operator_callback}
\begin{boxwithtitle}{\texttt{adj_register_operator_callback}}
\begin{minipage}{\columnwidth}
\begin{ccode}
int adj_register_operator_callback(adj_adjointer* adjointer, int type,
                                   char* name, void (*fn)(void));
\end{ccode}
\begin{fortrancode}
  function adj_register_operator_callback(adjointer, type, name, fnptr) 
           result(ierr)
    type(adj_adjointer), intent(inout) :: adjointer
    integer(kind=c_int), intent(in) :: type
    character(len=*), intent(in) :: name
    type(c_funptr), intent(in), value :: fnptr
    integer(kind=c_int) :: ierr
  end function adj_register_operator_callback
\end{fortrancode}
\end{minipage}
\end{boxwithtitle}
This function registers a given operator callback with blocks of name \texttt{name}
so that \libadjoint can use it when necessary. For a list of the valid values of
\texttt{type}, see \autoref{sec:operator_callback_types}. If an operator callback
has not been registered, and \libadjoint needs it to perform some task that the model
developer has requested, the function will return an \refapi{ADJ_ERR_NEED_CALLBACK}
error.

Example usage in C:

\begin{boxwithtitle}{\texttt{adj_register_operator_callback} (C example)}
\begin{minipage}{\columnwidth}
\begin{ccode}
  ierr = adj_register_operator_callback(adjointer, ADJ_NBLOCK_ACTION_CB, 
                                   "AdvectionOperator",
                                   (void (*)(void)) advection_action_proc);
  adj_chkierr(ierr);
\end{ccode}
\end{minipage}
\end{boxwithtitle}

Example usage in Fortran:

\begin{boxwithtitle}{\texttt{adj_register_operator_callback} (Fortran example)}
\begin{minipage}{\columnwidth}
\begin{fortrancode}
  ierr = adj_register_operator_callback(adjointer, ADJ_BLOCK_ASSEMBLY_CB, 
                                    "MassMatrix",
                                    c_funloc(mass_matrix_assembly_proc));
  call adj_chkierr(ierr);
\end{fortrancode}
\end{minipage}
\end{boxwithtitle}

As \libadjoint is only given the function pointer to the operator callback, it cannot check that the
interface of the given function is correct. If you register a function which is declared to take 
in an interface different to that which \libadjoint expects, then the model will segfault when
\libadjoint calls that function (or worse, smash the stack). Be very careful when writing operator callbacks to use exactly
the interface documented in this manual!
\subsection{Operator callback types} \label{sec:operator_callback_types}
\defapiss{ADJ_BLOCK_ACTION_CB}
\begin{boxwithtitle}{\texttt{ADJ_BLOCK_ACTION_CB}}
\begin{minipage}{\columnwidth}
\begin{ccode}
  void block_action(int ndepends, adj_variable* variables, adj_vector* dependencies,
                    int hermitian, adj_scalar coefficient, adj_vector input,
                    void* context, adj_vector* output);
\end{ccode}
\begin{fortrancode}
  subroutine block_action(ndepends, variables, dependencies, hermitian, coefficient,
                          input, context, output) 
    integer(kind=c_int), intent(in), value :: ndepends
    type(adj_variable), dimension(ndepends), intent(in) :: variables
    type(adj_vector), dimension(ndepends), intent(in) :: dependencies
    integer(kind=c_int), intent(in), value :: hermitian
    adj_scalar_f, intent(in), value :: coefficient
    type(adj_vector), intent(in), value :: input
    type(c_ptr), intent(in), value :: context
    type(adj_vector), intent(out) :: output
  end subroutine block_action
\end{fortrancode}
\end{minipage}
\end{boxwithtitle}
This callback computes the action of a block on a specified input vector \texttt{input}.

\begin{boxwithtitle}{Mathematical description (\texttt{ADJ_BLOCK_ACTION_CB})}
Let $V$ represent the block with which this routine is associated. Then,
if \texttt{hermitian == ADJ_FALSE}, this routine must compute
\begin{equation*}
\texttt{output} = \texttt{coefficient} \cdot V(d_1,d_2,\dots,d_{\texttt{ndepends}}) \cdot \texttt{input},
\end{equation*}
and if \texttt{hermitian == ADJ_TRUE}, this routine must compute
\begin{equation*}
\texttt{output} = \texttt{coefficient} \cdot V(d_1,d_2,\dots,d_{\texttt{ndepends}})^* \cdot \texttt{input},
\end{equation*}
where $(d_1, d_2, \dots, d_{\texttt{ndepends}})$ are the nonlinear dependencies of this block supplied in the \texttt{dependencies} array.
\end{boxwithtitle}

\begin{boxwithtitle}{Detailed argument list (\texttt{ADJ_BLOCK_ACTION_CB})}
\begin{description}
\item[\texttt{ndepends}] The number of variables that this block depends on.
\item[\texttt{variables}] An array of \refapi{adj_variable}s indicating which dependencies are being supplied to this routine. The length of the
array is given by \texttt{ndepends}. The first entry of the \texttt{variables} array
tells the model developer what the first entry of the \texttt{dependencies} is, the second pairs with the second, and so on.
\item[\texttt{dependencies}] An array of \refapi{adj_vector}s containing the values of the dependencies of this block. The dependencies supplied
here are the same as those which were registered during the annotation.
\item[\texttt{hermitian}] A flag indicating whether the routine is to compute the action of the block, or the action of the Hermitian of the
block. It will either be \texttt{ADJ_TRUE} or \texttt{ADJ_FALSE}.
\item[\texttt{coefficient}] A coefficient by which the routine must scale the output.
\item[\texttt{input}] The \refapi{adj_vector} to which the operator should be applied.
\item[\texttt{context}] This is a pointer to arbitrary model data, supplied at the time that the block was created during the annotation.
\item[\texttt{output}] This is an \refapi{adj_vector} which is to contain the output of the matrix-vector action.
\end{description}
\end{boxwithtitle}

When necessary: this callback is necessary for all blocks that appear off the diagonal of the annotation; i.e., the block
targets an \refapi{adj_variable} in an \refapi{adj_equation} that is not the variable being solved for in that
\refapi{adj_equation}.

\defapiss{ADJ_BLOCK_ASSEMBLY_CB}
\begin{boxwithtitle}{\texttt{ADJ_BLOCK_ASSEMBLY_CB}}
\begin{minipage}{\columnwidth}
\begin{ccode}
  void block_assembly(int ndepends, adj_variable* variables, adj_vector* dependencies,
                    int hermitian, adj_scalar coefficient, void* context,
                    adj_matrix* output, adj_vector* rhs);
\end{ccode}
\begin{fortrancode}
  subroutine block_assembly(ndepends, variables, dependencies, hermitian, coefficient,
                            context, output, rhs) 
    integer(kind=c_int), intent(in), value :: ndepends
    type(adj_variable), dimension(ndepends), intent(in) :: variables
    type(adj_vector), dimension(ndepends), intent(in) :: dependencies
    integer(kind=c_int), intent(in), value :: hermitian
    adj_scalar_f, intent(in), value :: coefficient
    type(c_ptr), intent(in), value :: context
    type(adj_matrix), intent(out) :: output
    type(adj_vector), intent(out) :: rhs
  end subroutine block_assembly
\end{fortrancode}
\end{minipage}
\end{boxwithtitle}
This callback assembles a given block, and also creates a model vector for its output.

\begin{boxwithtitle}{Mathematical description (\texttt{ADJ_BLOCK_ASSEMBLY_CB})}
Let $V$ represent the block with which this routine is associated. Then,
if \texttt{hermitian == ADJ_FALSE}, this routine must compute
\begin{equation*}
\texttt{output} = \texttt{coefficient} \cdot V(d_1,d_2,\dots,d_{\texttt{ndepends}}),
\end{equation*}
and if \texttt{hermitian == ADJ_TRUE}, this routine must compute
\begin{equation*}
\texttt{output} = \texttt{coefficient} \cdot V(d_1,d_2,\dots,d_{\texttt{ndepends}})^*,
\end{equation*}
where $(d_1, d_2, \dots, d_{\texttt{ndepends}})$ are the nonlinear dependencies of this block supplied in the \texttt{dependencies} array.
\end{boxwithtitle}

\begin{boxwithtitle}{Detailed argument list (\texttt{ADJ_BLOCK_ASSEMBLY_CB})}
\begin{description}
\item[\texttt{ndepends}] The number of variables that this block depends on.
\item[\texttt{variables}] An array of \refapi{adj_variable}s indicating which dependencies are being supplied to this routine. The length of the
array is given by \texttt{ndepends}. The first entry of the \texttt{variables} array
tells the model developer what the first entry of the \texttt{dependencies} is, the second pairs with the second, and so on.
\item[\texttt{dependencies}] An array of \refapi{adj_vector}s containing the values of the dependencies of this block. The dependencies supplied
here are the same as those which were registered during the annotation.
\item[\texttt{hermitian}] A flag indicating whether the routine is to compute the action of the block, or the action of the Hermitian of the
block. It will either be \texttt{ADJ_TRUE} or \texttt{ADJ_FALSE}.
\item[\texttt{coefficient}] A coefficient by which the routine must scale the output.
\item[\texttt{context}] This is a pointer to arbitrary model data, supplied at the time that the block was created during the annotation.
\item[\texttt{output}] This is an \refapi{adj_matrix} which is to contain the matrix assembled at these dependencies.
\item[\texttt{rhs}] This is a model \refapi{adj_vector} which lies in the output space of the operator to be assembled. 
\end{description}
\end{boxwithtitle}

When necessary: this callback is necessary for all blocks that appear on the diagonal of the annotation; i.e., the block
targets an \refapi{adj_variable} in an \refapi{adj_equation} that is the variable being solved for in that
\refapi{adj_equation}.

\defapiss{ADJ_NBLOCK_DERIVATIVE_ACTION_CB}
\begin{boxwithtitle}{\texttt{ADJ_NBLOCK_DERIVATIVE_ACTION_CB}}
\begin{minipage}{\columnwidth}
\begin{ccode}
void nblock_derivative_action(int ndepends, adj_variable* variables, 
                              adj_vector* dependencies, adj_variable derivative, 
                              adj_vector contraction, int hermitian, 
                              adj_vector input, adj_scalar coefficient, 
                              void* context, adj_vector* output);
\end{ccode}
\begin{fortrancode}
subroutine nblock_derivative_action(ndepends, variables, dependencies, derivative, 
                                    contraction, hermitian, input, coefficient, 
                                    context, output) 
  integer(kind=c_int), intent(in), value :: ndepends
  type(adj_variable), dimension(ndepends), intent(in) :: variables
  type(adj_vector), dimension(ndepends), intent(in) :: dependencies
  type(adj_variable), intent(in), value :: derivative
  type(adj_vector), intent(in), value :: contraction
  integer(kind=c_int), intent(in), value :: hermitian
  type(adj_vector), intent(in), value :: input
  adj_scalar_f, intent(in), value :: coefficient
  type(c_ptr), intent(in), value :: context
  type(adj_vector), intent(out) :: output
end subroutine nblock_derivative_action
\end{fortrancode}
\end{minipage}
\end{boxwithtitle}
This callback computes the the derivative of a given nonlinear block with respect \texttt{derivative} and contracted with \texttt{contraction}. 
The action of this derivative on a specified input vector \texttt{input} is returned.

\begin{boxwithtitle}{Mathematical description (\texttt{ADJ_NBLOCK_DERIVATIVE_ACTION_CB})}
Let $V$ represent the nonlinear block with which this routine is associated. 
Suppose $V$ depends on $\left\{d_0, d_1, \dots, d_\texttt{ndepends}\right\}$. The values of these                                                  
dependencies are given by the \texttt{dependencies} array, with each corresponding entry in the \texttt{variables} array describing what each variable is. \texttt{derivative} 
is the variable with which the derivative is to be computed. Let $d_i$ denote this variable, and $c$ denote \texttt{contraction}.
Then, if \texttt{hermitian == ADJ_FALSE}, this routine must compute
\begin{equation*}
\texttt{output} = \left(\left.\frac{\partial V}{\partial d_i}\right|_{(d_0, \dots, d_{\texttt{ndepends}})}c\right)\texttt{input}, 
\end{equation*}
and if \texttt{hermitian == ADJ_TRUE}, this routine must compute
\begin{equation*}
\texttt{output} = \left(\left.\frac{\partial V}{\partial d_i}\right|_{(d_0, \dots, d_{\texttt{ndepends}})}c\right)^{*}\texttt{input}. 
\end{equation*}
\end{boxwithtitle}

\begin{boxwithtitle}{Using automatic differentiation tools to generate \texttt{ADJ_NBLOCK_DERIVATIVE_ACTION_CB} callbacks.}
The \texttt{ADJ_NBLOCK_DERIVATIVE_ACTION_CB} callback function implements the derivatives of the nonlinear block with respect to its dependencies.
While these derivatives can be derived by hand for simple nonlinear blocks, it is desirable to automatically generate the callbacks with an automatic differentiation (AD) tool.
The following example demonstrates a typical setup of this kind. 

Assume the associated block computes the action with the advection matrix $A(u)c$,
where $u$ is the (only) dependency of the nonlinear block $A$ and $c$ is the contraction vector.
A helper subroutine is available that performs this action:

\begin{minipage}{\columnwidth}
\begin{fortrancode}
subroutine advection_action(u, c, Ac)
  real, dimension(:), intent(in) :: u
  real, dimension(:), intent(in) :: c
  real, dimension(:), intent(out) :: Ac
...
end subroutine advection_action
\end{fortrancode}
\end{minipage}

From applying the AD tool in forward mode and instructing it to generate the derivative of \texttt{Ac} with respect to \texttt{u} one obtains:

\begin{minipage}{\columnwidth}
\begin{fortrancode}
! Differentiation of advection_action in forward (tangent) mode:
! variations of useful results: Ac
! with respect to varying inputs: u
! RW status of diff variables: Ac:out u:in
subroutine advection_action_d(u, ud, c, Ac, Acd)
  real, dimension(:), intent(in) :: u
  real, dimension(:), intent(in) :: ud
  real, dimension(:), intent(in) :: c
  real, dimension(:), intent(out) :: Ac
  real, dimension(:), intent(out) :: Acd
...
end subroutine advection_action_d
\end{fortrancode}
\end{minipage}

and 

\begin{minipage}{\columnwidth}
\begin{fortrancode}
! Differentiation of advection_action in reverse (adjoint) mode:
! gradient of useful results: Ac
! with respect to varying inputs: Ac u
! RW status of diff variables: Ac:in-zero u:out
subroutine advection_action_b(u, ub, c, Ac, Acb)
    real, dimension(:), intent(in) :: u
    real, dimension(:), intent(out) :: ub
    real, dimension(:), intent(in) :: c
    real, dimension(:), intent(out) :: Ac
    real, dimension(:), intent(in) :: Acb
...
end subroutine advection_action_b
\end{fortrancode}
\end{minipage}

These routines can now be used in the \texttt{ADJ_NBLOCK_DERIVATIVE_ACTION_CB} callback:

\begin{minipage}{\columnwidth}
\begin{fortrancode}
subroutine advection_derivative_action(ndepends, variables, dependencies, 
                                       derivative, contraction, hermitian, 
                                       input, coefficient, context, output)
...
if (hermitian == ADJ_FALSE) then
  call advection_action_d(dependencies(1), input, contraction, tmp, output)
else
  call advection_action_b(dependencies(1), output, contraction, tmp, input)
end if
...
end subroutine advection_derivative_action
\end{fortrancode}
\end{minipage}


Note that most AD tools only support a subset of the programming language. 
While this can be a burden when it is used to generate the adjoint of the full forward model,
in the context of libadjoint it is only applied to the assembly routine of the nonlinear block, which normally consists only of low level mathematical routines.
 
\end{boxwithtitle}

\begin{boxwithtitle}{Detailed argument list (\texttt{ADJ_NBLOCK_DERIVATIVE_ACTION_CB})}
\begin{description}
\item[\texttt{ndepends}] The number of variables that the associated nonlinear block depends on.
\item[\texttt{variables}] An array of \refapi{adj_variable}s indicating which dependencies are being supplied to this routine. The length of the
array is given by \texttt{ndepends}. The first entry of the \texttt{variables} array
tells the model developer what the first entry of the \texttt{dependencies} is, the second pairs with the second, and so on.
\item[\texttt{dependencies}] An array of \refapi{adj_vector}s containing the values of the dependencies of this block. The dependencies supplied
here are the same as those which were registered during the annotation.
\item[\texttt{derivative}] The derivative of the nonlinear block is to be computed with respect to this \refapi{adj_variable}. 
\item[\texttt{contraction}] The \refapi{adj_vector} with which the derivative is to be contracted.  
\item[\texttt{hermitian}] A flag indicating whether the routine is to compute the action of the block, or the action of the Hermitian of the
block. It will either be \texttt{ADJ_TRUE} or \texttt{ADJ_FALSE}.
\item[\texttt{input}] The \refapi{adj_vector} to which the operator should be applied.
\item[\texttt{coefficient}] A coefficient by which the routine must scale the output.
\item[\texttt{context}] This is a pointer to arbitrary model data, supplied at the time that the block was created during the annotation.
\item[\texttt{output}] This is an \refapi{adj_vector} which is to contain the output of the matrix-vector action.
\end{description}
\end{boxwithtitle}

When necessary: required for each registered nonlinear block for assembling the adjoint equation, see \autoref{sec:adjoint_assembly}.

%\defapiss{ADJ_NBLOCK_DERIVATIVE_ASSEMBLY_CB}
%\begin{boxwithtitle}{\texttt{ADJ_NBLOCK_DERIVATIVE_ASSEMBLY_CB}}
%\begin{minipage}{\columnwidth}
%\begin{ccode}
%void nblock_derivative_assembly(int ndepends, adj_variable* variables, 
%                                adj_vector* dependencies, adj_variable derivative, 
%                                adj_vector contraction, int hermitian, 
%                                adj_scalar coefficient, 
%                                void* context, adj_vector* output);
%\end{ccode}
%\begin{fortrancode}
%subroutine nblock_derivative_assembly(ndepends, variables, dependencies, derivative, 
%                                      contraction, hermitian, coefficient, 
%                                      context, output) 
%  integer(kind=c_int), intent(in) :: ndepends
%  type(adj_variable), dimension(ndepends), intent(in) :: variables
%  type(adj_vector), dimension(ndepends), intent(in) :: dependencies
%  type(adj_variable), intent(in) :: derivative
%  type(adj_vector), intent(in) :: contraction
%  logical(kind=c_bool), intent(in) :: hermitian
%  type(c_ptr), intent(in) :: context
%  type(adj_matrix), intent(out) :: output
%end subroutine nblock_derivative_assembly
%\end{fortrancode}
%\end{minipage}
%\end{boxwithtitle}
%This callback assembles the derivative of a given nonlinear block with respect \texttt{derivative} and contracted with \texttt{contraction}.
%
%
%\begin{boxwithtitle}{Mathematical description (\texttt{ADJ_NBLOCK_DERIVATIVE_ASSEMBLY_CB})}
%Let $V$ represent the nonlinear block with which this routine is associated. 
%Suppose $V$ depends on $\left\{d_0, d_1, \dots, d_N\right\}$. The values of these                                                  
%dependencies are given by the \texttt{dependencies} array, with each corresponding entry in the \texttt{variables} array describing what each variable is. \texttt{derivative} 
%is the variable with which the derivative is to be computed. Let $d_i$ denote this variable, and $c$ denote \texttt{contraction}.
%Then, if \texttt{hermitian == ADJ_FALSE}, this routine must compute
%\begin{equation*}
%\texttt{output} = \left(\left.\frac{\partial V}{\partial d_i}\right|_{(d_0, \dots, d_{\texttt{ndepends}})}c\right),
%\end{equation*}
%and if \texttt{hermitian == ADJ_TRUE}, this routine must compute
%\begin{equation*}
%\texttt{output} = \left(\left.\frac{\partial V}{\partial d_i}\right|_{(d_0, \dots, d_{\texttt{ndepends}})}c\right)^{*},
%\end{equation*}
%\end{boxwithtitle}
%
%\begin{boxwithtitle}{Detailed argument list (\texttt{ADJ_NBLOCK_DERIVATIVE_ASSEMBLY_CB})}
%\begin{description}
%\item[\texttt{ndepends}] The number of variables that the associated nonlinear block depends on.
%\item[\texttt{variables}] An array of \refapi{adj_variable}s indicating which dependencies are being supplied to this routine. The length of the
%array is given by \texttt{ndepends}. The first entry of the \texttt{variables} array
%tells the model developer what the first entry of the \texttt{dependencies} is, the second pairs with the second, and so on.
%\item[\texttt{dependencies}] An array of \refapi{adj_vector}s containing the values of the dependencies of this block. The dependencies supplied
%here are the same as those which were registered during the annotation.
%\item[\texttt{derivative}] The \refapi{adj_variable} with which respect to the derivative is to be computed.
%\item[\texttt{contraction}] The \refapi{adj_vector} with which the derivative is to be contracted.  
%\item[\texttt{hermitian}] A flag indicating whether the routine is to compute the action of the block, or the action of the Hermitian of the
%block. It will either be \texttt{ADJ_TRUE} or \texttt{ADJ_FALSE}.
%\item[\texttt{input}] The \refapi{adj_vector} to which the operator should be applied.
%\item[\texttt{coefficient}] A coefficient by which the routine must scale the output.
%\item[\texttt{context}] This is a pointer to arbitrary model data, supplied at the time that the block was created during the annotation.
%\item[\texttt{output}] This is an \refapi{adj_matrix} which is to contain the derivative matrix.
%\end{description}
%\end{boxwithtitle}
%
%When necessary: 
%sometimes required for a nonlinear block registered when the
%ADJ DIFFERENTIATION option is set to ADJ DIFFERENTIATION SUPPLIED. If a nonlinear
%block depends on variables computed earlier in the simulation, then only the action is
%required; if a nonlinear block depends on the variable being solved for in that equation,
%then the assembly is required.

\defapiss{ADJ_NBLOCK_ACTION_CB}
\begin{boxwithtitle}{\texttt{ADJ_NBLOCK_ACTION_CB}}
\begin{minipage}{\columnwidth}
\begin{ccode}
void nblock_action(int ndepends, adj_variable* variables, adj_vector* dependencies, 
                 adj_vector input, void* context, adj_vector* output);
\end{ccode}
\begin{fortrancode}
subroutine nblock_action(ndepends, variables, dependencies, input, 
                         context, output) 
  integer(kind=c_int), intent(in), value :: ndepends
  type(adj_variable), dimension(ndepends), intent(in) :: variables
  type(adj_vector), dimension(ndepends), intent(in) :: dependencies
  type(adj_vector), intent(in), value :: input
  type(c_ptr), intent(in), value :: context
  type(adj_vector), intent(out) :: output
end subroutine nblock_action
\end{fortrancode}
\end{minipage}
\end{boxwithtitle}

This callback computes the action of a nonlinear block on a specified input vector \texttt{input}.

\begin{boxwithtitle}{Mathematical description (\texttt{ADJ_NBLOCK_ACTION_CB})}
Let $V$ represent the nonlinear block with which this routine is associated. 
Then, this routine must compute
\begin{equation*}
\texttt{output} = \texttt{coefficient} \cdot V(d_1,d_2,\dots,d_{\texttt{ndepends}}) \cdot \texttt{input},
\end{equation*}
where $(d_1, d_2, \dots, d_{\texttt{ndepends}})$ are the nonlinear dependencies of this block supplied in the \texttt{dependencies} array.
\end{boxwithtitle}

\begin{boxwithtitle}{Detailed argument list (\texttt{ADJ_NBLOCK_ACTION_CB})}
\begin{description}
\item[\texttt{ndepends}] The number of variables that this nonlinear block depends on.
\item[\texttt{variables}] An array of \refapi{adj_variable}s indicating which dependencies are being supplied to this routine. The length of the
array is given by \texttt{ndepends}. The first entry of the \texttt{variables} array
tells the model developer what the first entry of the \texttt{dependencies} is, the second pairs with the second, and so on.
\item[\texttt{dependencies}] An array of \refapi{adj_vector}s containing the values of the dependencies of this nonlinear block. The dependencies supplied
here are the same as those which were registered during the annotation.
\item[\texttt{input}] The \refapi{adj_vector} to which the operator should be applied.
\item[\texttt{context}] This is a pointer to arbitrary model data, supplied at the time that the nonlinear block was created during the annotation.
\item[\texttt{output}] This is an \refapi{adj_vector} which is to contain the output of the matrix-vector action.
\end{description}
\end{boxwithtitle}

When necessary: 
This callback is necessary for checking the gradient correctness, see \autoref{sec:derivative_test}.

\section{Source-term callbacks}
\defapis{adj_register_forward_source_callback}
\begin{boxwithtitle}{\texttt{adj_register_forward_source_callback}}
\begin{minipage}{\columnwidth}
\begin{ccode}
int adj_register_forward_source_callback(adj_adjointer* adjointer, 
                 void (*fn)(adj_adjointer* adjointer, adj_variable variable, 
                            int ndepends, adj_variable* variables, 
                            adj_vector* dependencies, void* context, 
                            adj_vector* output, int* has_output));
\end{ccode}
\begin{fortrancode}
function adj_register_forward_source_callback(adjointer, fnptr) 
                                                   result(ierr) 
  type(adj_adjointer), intent(inout) :: adjointer
  type(c_funptr), intent(in), value :: fnptr
  integer(kind=c_int) :: ierr
end function adj_register_forward_source_callback
\end{fortrancode}
\end{minipage}
\end{boxwithtitle}

The supplied function callback computes the source vector when solving for \texttt{variable} in the forward model.
Its interface has to conform:

\begin{boxwithtitle}{Function interface for \texttt{adj_register_forward_source_callback}}
\begin{minipage}{\columnwidth}
\begin{ccode}
void forward_source(adj_adjointer* adjointer, adj_variable variable, 
                    int ndepends, adj_variable* variables, 
                    adj_vector* dependencies, void* context, 
                    adj_vector* output, int* has_output);
\end{ccode}
\begin{fortrancode}
subroutine forward_source(adjointer, variable, ndepends, dependencies, 
                          values, output, has_output) 
  type(adj_adjointer), intent(in) :: adjointer
  type(adj_variable), intent(in), value :: variable
  integer(kind=c_int), intent(in), value :: ndepends
  type(adj_variable), dimension(ndepends), intent(in) :: dependencies
  type(adj_vector), dimension(ndepends), intent(in) :: values
  type(adj_vector), intent(out) :: output
  integer(kind=c_int), intent(out) :: has_output
end subroutine forward_source
\end{fortrancode}
\end{minipage}
\end{boxwithtitle}


\begin{boxwithtitle}{Detailed argument list (\texttt{forward_source})}
\begin{description}
\item[\texttt{adjointer}] The associated \refapi{adj_adjointer}.
\item[\texttt{ndepends}] The number of variables that this nonlinear block depends on.
\item[\texttt{variables}] An array of \refapi{adj_variable}s indicating which dependencies are being supplied to this routine. The length of the
array is given by \texttt{ndepends}. The first entry of the \texttt{variables} array
tells the model developer what the first entry of the \texttt{dependencies} is, the second pairs with the second, and so on.
\item[\texttt{dependencies}] An array of \refapi{adj_vector}s containing the values of the dependencies of this nonlinear block. The dependencies supplied
here are the same as those which were registered during the annotation.
\item[\texttt{context}] This is a pointer to arbitrary model data, supplied at the time that the nonlinear block was created during the annotation.
\item[\texttt{output}] If \texttt{has_output == ADJ_true}, this is an \refapi{adj_vector} which is to contain the output of the matrix-vector action. Otherwise \texttt{output} can be left unallocated.
\item[\texttt{has_output}] Must be set to \texttt{ADJ_FALSE} if no source is apparent. Otherwise, it must be set to \texttt{ADJ_TRUE}.
\end{description}
\end{boxwithtitle}

When necessary: 
This callback is necessary when the annotated model is compared with the original run, see \autoref{sec:compare}.

\section{Functional evaluation callbacks}
\defapis{adj_register_functional_callback}

\begin{boxwithtitle}{\texttt{adj_register_functional_callback}}
\begin{minipage}{\columnwidth}
\begin{ccode}
int adj_register_functional_callback(adj_adjointer* adjointer, char* name, 
                           void (*fn)(adj_adjointer* adjointer, int timestep, 
                                      int ndepends, adj_variable* variables, 
                                      adj_vector* dependencies, char* name, 
                                      adj_scalar* output));
\end{ccode}
\begin{fortrancode}
  function adj_register_functional_callback(adjointer, name, fnptr) result(ierr)
    type(adj_adjointer), intent(inout) :: adjointer
    character(len=*), intent(in) :: name
    type(c_funptr), intent(in) :: fnptr
    integer :: ierr
\end{fortrancode}
\end{minipage}
\end{boxwithtitle}

The supplied function callback computes the functional value at \texttt{timestep} of the functional with name \texttt{name}.
Its interface has to conform:

\begin{boxwithtitle}{Function interface for \texttt{adj_register_functional_callback}}
\begin{minipage}{\columnwidth}
\begin{ccode}
void functional(adj_adjointer* adjointer, int timestep, 
                         int ndepends, adj_variable* variables, 
                         adj_vector* dependencies, char* name, 
                         adj_scalar* output);
\end{ccode}
\begin{fortrancode}
subroutine functional(adjointer, timestep, ndepends, variables, dependencies, 
                      name, output) 
  type(adj_adjointer), intent(in) :: adjointer
  integer(kind=c_int), intent(in), value :: timestep
  integer(kind=c_int), intent(in), value :: ndepends
  type(adj_variable), dimension(ndepends), intent(in) :: variables 
  type(adj_vector), dimension(ndepends), intent(in) :: dependencies
  character(kind=c_char), dimension(ADJ_NAME_LEN), intent(in) :: name
  adj_scalar_f, intent(out) :: output
end subroutine functional
\end{fortrancode}
\end{minipage}
\end{boxwithtitle}

\begin{boxwithtitle}{Detailed argument list (\texttt{functional})}
\begin{description}
\item[\texttt{adjointer}] The associated \refapi{adj_adjointer}.
\item[\texttt{timestep}] The timestep at which the functional is to be evaluated. The start and end time of this timestep can be obtained with \texttt{adj_timestep_get_times}.
\item[\texttt{ndepends}] The number of variables that this functional depends on.
\item[\texttt{variables}] An array of \refapi{adj_variable}s indicating which dependencies are being supplied to this routine. The length of the
array is given by \texttt{ndepends}. The first entry of the \texttt{variables} array
tells the model developer what the first entry of the \texttt{dependencies} is, the second pairs with the second, and so on.
\item[\texttt{dependencies}] An array of \refapi{adj_vector}s containing the values of the dependencies of this functional. The dependencies supplied
here are the same as those which were registered during the annotation.
\item[\texttt{name}] The name of the functional to be evaluated.
\item[\texttt{output}] This is an \refapi{adj_vector} which is to contain the functional value. 
\end{description}
\end{boxwithtitle}

\defapis{adj_register_functional_derivative_callback}

\begin{boxwithtitle}{\texttt{adj_register_functional_derivative_callback}}
\begin{minipage}{\columnwidth}
\begin{ccode}
int adj_register_functional_derivative_callback(
                    adj_adjointer* adjointer, char* name, 
                    void (*fn)(adj_adjointer* adjointer, 
                               adj_variable variable, int ndepends, 
                               adj_variable* variables, adj_vector* dependencies, 
                               char* name, adj_vector* output));
\end{ccode}
\begin{fortrancode}
function adj_register_functional_derivative_callback(adjointer, name, fnptr) 
                                                                      result(ierr)
  type(adj_adjointer), intent(inout) :: adjointer
  character(len=*), intent(in) :: name
  type(c_funptr), intent(in) :: fnptr
  integer :: ierr
end function adj_register_functional_derivative_callback
\end{fortrancode}
\end{minipage}
\end{boxwithtitle}

The supplied function callback computes the derivative of the functional with respect to the variable \texttt{derivative}. 
Its interface has to conform:

\begin{boxwithtitle}{\texttt{Function interface for \texttt{adj_register_functional_derivative_callback}}}
\begin{minipage}{\columnwidth}
\begin{ccode}
void functional_derivative(adj_adjointer* adjointer,
                               adj_variable derivative, int ndepends, 
                               adj_variable* variables, adj_vector* dependencies,
                               char* name, adj_vector* output);
\end{ccode}
\begin{fortrancode}
subroutine functional_derivative(adjointer, derivative, ndepends, variables, 
                                 dependencies, name, output) 
  type(adj_adjointer), intent(in) :: adjointer
  type(adj_variable), intent(in), value :: derivative 
  integer(kind=c_int), intent(in), value :: ndepends
  type(adj_variable), dimension(ndepends), intent(in) :: variables
  type(adj_vector), dimension(ndepends), intent(in) :: dependencies
  character(kind=c_char), dimension(ADJ_NAME_LEN), intent(in) :: name
  type(adj_vector), intent(out) :: output
end subroutine functional_derivative
\end{fortrancode}
\end{minipage}
\end{boxwithtitle}

\begin{boxwithtitle}{Detailed argument list (\texttt{functional_derivative})}
\begin{description}
\item[\texttt{adjointer}] The associated \refapi{adj_adjointer}.
\item[\texttt{derivative}] The functional derivative is to be computed with respect to this \refapi{adj_variable}. The timestep and iteration number of this variable can be obtained with \refapi{adj_variable_get_timestep} and \refapi{adj_variable_get_iteration}.
\item[\texttt{ndepends}] The number of variables that this functional depends on.
\item[\texttt{variables}] An array of \refapi{adj_variable}s indicating which dependencies are being supplied to this routine. The length of the
array is given by \texttt{ndepends}. The first entry of the \texttt{variables} array
tells the model developer what the first entry of the \texttt{dependencies} is, the second pairs with the second, and so on.
\item[\texttt{dependencies}] An array of \refapi{adj_vector}s containing the values of the dependencies of this nonlinear block. The dependencies supplied
here are the same as those which were registered during the annotation.
\item[\texttt{name}] The name of the functional of which the derivative is to be computed.
\item[\texttt{output}] This is an \refapi{adj_vector} which is to contain the derivative of the functional with respect to \texttt{derivative}. 
\end{description}
\end{boxwithtitle}
