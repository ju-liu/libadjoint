\chapter{Annotating forward models}

\begin{synopsis}
\end{synopsis}
\minitoc
\vspace{\fill}
\newpage

\section{A simple example} \label{sec:examples}
\section{Existing model code}
\section{Annotating the model}
\defapis{adj_timestep_set_times}
\begin{boxwithtitle}{Function interface for \texttt{adj_timestep_set_times}}
\begin{minipage}{\columnwidth}
\begin{ccode}
  int adj_timestep_set_times(adj_adjointer* adjointer, int timestep, 
                             adj_scalar start, adj_scalar end)
\end{ccode}
\begin{fortrancode}
  function adj_timestep_set_times(adjointer, timestep, start, end) result(ierr) 
    type(adj_adjointer), intent(inout) :: adjointer
    integer(kind=c_int), intent(in), value :: timestep
    adj_scalar_f, intent(in), value :: start
    adj_scalar_f, intent(in), value :: end
    integer(kind=c_int) :: ierr
  end function adj_timestep_set_times
\end{fortrancode}
\end{minipage}
\end{boxwithtitle}

This functions sets the simulation start and end time for time step \texttt{timestep} in the \texttt{adjointer} object.
The function \refapi{adj_timestep_get_times} can be used to retrieve these information again. 

When necessary: Providing the simulation times is optional and is only used by \refapi{adj_timestep_get_times}.

\defapis{adj_timestep_set_functional_dependencies}
\begin{boxwithtitle}{Function interface for \texttt{adj_timestep_set_functional_dependencies}}
\begin{minipage}{\columnwidth}
\begin{ccode}
  int adj_timestep_set_functional_dependencies(adj_adjointer* adjointer, 
                   int timestep, char* functional, int ndepends, 
                   adj_variable* dependencies);
\end{ccode}
\begin{fortrancode}
  function adj_timestep_set_functional_dependencies(adjointer, timestep, 
                                  functional, dependencies) result(ierr)
    type(adj_adjointer), intent(inout) :: adjointer
    integer, intent(in) :: timestep
    character(len=*), intent(in) :: functional
    type(adj_variable), dimension(:), intent(in) :: dependencies
    integer :: ierr
  end function adj_timestep_set_functional_dependencies
\end{fortrancode}
\end{minipage}
\end{boxwithtitle}

This function defines the dependencies for time step \texttt{timestep} of the functional with name \texttt{functional}.
The argument \texttt{dependencies} must be an array of length \texttt{ndepends} containing any variable that is needed to evaluate the functional associated with this time step.

More specifically, \libadjoint assumes that the functional $J$ is written as a sum over all time steps:
\begin{equation}
J = \sum_{n=0}^N J_n(\mathbf d_n),
\label{eq:functional_as_sum}
\end{equation}
where $N$ is the number of time steps and $\mathbf d_n=\{d_n^1, d_n^2, ...\}$ are the dependencies for the n'th summand.
Then $\mathbf d_n$ is the \texttt{dependencies} argument of  \refapi{adj_timestep_set_functional_dependencies} at time step $i$.

This information is always required and used in two ways: 
\begin{itemize}
\item Firstly, to work out the dependency variables passed to the functional evaluation callback, see \refapi{adj_register_functional_callback}.
This callback function then computes the associated summand of this timestep, i.e. $J_n(\mathbf d_n)$ for time step $n$.
\item Secondly, to work out which dependency variables are needed for the evaluation of the functional derivatives, see \refapi{adj_register_functional_derivative_callback}.
\end{itemize}

Examples:

\underline{Example 1}:
A common functional is the misfit of the forward solution $u$ and measurements $w$ at a given time $T$, i.e.:

\begin{equation}
J(u) = ||u(T) - w(T)||.
\label{eq:ex_functional1}
\end{equation}

Since time $T$ does not have to coincide with one of the simulation time steps, the solution $u$ needs to be interpolated at time $T$.
Let $u^{n-1}$ and $u^{n}$ be the two solution vectors at the time steps $t^{n-1}$ and $t^{n}$ just before and after $T$.
Then a linear interpolation yields:
\begin{equation}
J(u) = ||\frac{T-t^{n}}{t^{n-1}-t^{n}}u^{n-1} + \frac{T-t^{n-1}}{t^{n}-t^{n-1}}u^{n} - w(T)||.
\label{eq:ex_functional2}
\end{equation}

Therefore there are two dependency variables, $u^{n-1}$ and $u^{n}$, for time step $n$ (or three if $w(T)$ is declared to be an auxiliary variable) and no dependencies for all other time steps.

\underline{Example 2}:
If measurements are available at any point in time the misfit function can be integrated over time, i.e.: 
\begin{equation}
J(u) = \int_0^T ||u(t) - w(t)||^2 dt,
\label{eq:ex_functional3}
\end{equation}

where $0$ and $T$ are the simulation start and end time respectively, $u$ is the forward solution and $w$ contains the measurement data.


To bring \autoref{eq:ex_functional3} in the form of \autoref{eq:functional_as_sum} it is rewritten as:
\begin{equation}
J(u) = \sum_{n=1,..,N} J_n(u^{n-1}, u^{n}), \quad J_n(u^{n-1}, u^{n}) := \int_{(n-1)\Delta t}^{n \Delta t} ||u - w||^2 dt,
\end{equation}
where $N$ is the number of time steps, $\Delta t$ is the time step size and a superscript denotes the time step of the variable.
The dependency variables for time step $n>0$ is therefore $u^{n-1}$ and $u^{n}$ (and the associated $w$ if declared as auxiliary variables).

\underline{Example 3}:
Consider the functional given by: 
\begin{equation}
J(u) = \prod_{n=0,..,N} u^{n}.
\end{equation}
This can be trivially rewritten in form \autoref{eq:functional_as_sum} by defining $J_n \equiv 0$ for $n \ne N$ and $J_N(u) = \prod_{n=0,..,N} u^{n}$.
Hence, there are no dependencies for time steps smaller than $n$ and a dependency of all forward variables at time step $N$.


