\chapter{Annotating forward models}

\begin{synopsis}
\end{synopsis}
\minitoc
\vspace{\fill}
\newpage

\section{A simple example} \label{sec:examples}
\section{Existing model code}
\section{Annotating the model}
\defapis{adj_timestep_set_times}
\begin{boxwithtitle}{Function interface for \texttt{adj_timestep_set_times}}
\begin{minipage}{\columnwidth}
\begin{ccode}
  int adj_timestep_set_times(adj_adjointer* adjointer, int timestep, 
                             adj_scalar start, adj_scalar end)
\end{ccode}
\begin{fortrancode}
  function adj_timestep_set_times(adjointer, timestep, start, end) result(ierr) 
    type(adj_adjointer), intent(inout) :: adjointer
    integer(kind=c_int), intent(in), value :: timestep
    adj_scalar_f, intent(in), value :: start
    adj_scalar_f, intent(in), value :: end
    integer(kind=c_int) :: ierr
  end function adj_timestep_set_times
\end{fortrancode}
\end{minipage}
\end{boxwithtitle}

When necessary:

\defapis{adj_timestep_set_functional_dependencies}
\begin{boxwithtitle}{Function interface for \texttt{adj_timestep_set_functional_dependencies}}
\begin{minipage}{\columnwidth}
\begin{ccode}
  int adj_timestep_set_functional_dependencies(adj_adjointer* adjointer, 
                   int timestep, char* functional, int ndepends, 
                   adj_variable* dependencies);
\end{ccode}
\begin{fortrancode}
  function adj_timestep_set_functional_dependencies(adjointer, timestep, 
                                  functional, dependencies) result(ierr)
    type(adj_adjointer), intent(inout) :: adjointer
    integer, intent(in) :: timestep
    character(len=*), intent(in) :: functional
    type(adj_variable), dimension(:), intent(in) :: dependencies
    integer :: ierr
  end function adj_timestep_set_functional_dependencies
\end{fortrancode}
\end{minipage}
\end{boxwithtitle}

This function defines the dependency variables for the functional with name \texttt{functional} at timestep \texttt{timestep}.
\texttt{dependencies} must be an array of length \texttt{ndepends} containing any variable that is needed to evaluate the functional at this timestep.

This information is used in two ways: 
Firstly, to work out the dependecy variables needed for the evaluation of the fucntional value, see \refapi{adj_register_functional_callback}.
Secondly, to work out the dependency variables needed for the evaluation of the functional derivatives, see \refapi{adj_register_functional_derivative_callback}.


Note 

A simpe example:
A very common functional is the misfit of the solution $u$ and measurements $u_m(T)$ at the end of the simlation time $T$:
\begin{equation}
J(u) = |u(T) - u_{m}(T)|^2
\label{eq:ex_functional1}
\end{equation}





\begin{equation}
J(u) = \int_0^T |u - u_{m}|^2 dt,
\label{eq:ex_functional1}
\end{equation}

where $0$ and $T$ are the simulation start end end time respectively, $u$ is the forward solution vector and $u_m$ is a vector containing available measurement data.


If $\Delta t$ denotes the timestep used for a linear time discretisation (for example the $\theta$ method) of the forward model, we can write \autoref{eq:ex_functional1} as:
\begin{equation}
J(u) = \sum_{i=0,..,T/\Delta t} \int_0^T |u - u_{m}|^2 dt
\label{eq:ex_functional1}
\end{equation}
where the superscript denotes the timelevel of the



When necessary: Always.


