\chapter{Callbacks} \label{chap:callbacks}

\begin{synopsis}
\end{synopsis}
\minitoc
\vspace{\fill}
\newpage

\section{Data callbacks}
\subsection{The \texttt{adj_vector} and \texttt{adj_matrix} types}
\libadjoint needs to manipulate vectors and matrices. After all, the entire
purpose of the main output function \refapi{adj_get_adjoint_equation} is to
assemble a left-hand-side matrix and a right-hand-side vector. Therefore,
\libadjoint needs to have classes representing the concepts of vectors and matrices.

However, one of the
design goals of \libadjoint is to be applicable to many different models,
using very different data structures. How, then, should the \libadjoint
vector and matrix classes work?

One approach is for \libadjoint to define its own vector and matrix classes,
to which the model developer must convert the model data structures. However,
writing yet another linear algebra API is highly undesirable; the world
has enough of these.

A second approach would be to standardise on some common linear algebra
API such as PETSc \citep{balay1997,balay2010} or Trilinos \citep{heroux2003}. However,
this would introduce a large hard dependency on such a library to any project which uses \libadjoint;
the aim for \libadjoint is to be small, lightweight, portable, and easy to install. Furthermore,
it would entail having the same data around in memory in two different formats (the format blessed by \libadjoint,
and the model's own), which would be very inefficient.

\libadjoint settles on a third approach, inspired by the example of the Zoltan
graph partitioning library \citep{devine2002}. The \libadjoint vector and matrix
classes are merely thin wrappers around the user's own data structures;
at their core, they are nothing but a void pointer. While this approach has the
great advantage that it can be used with any and all pre-existing data structures,
it means that \libadjoint has a problem: how can it manipulate these objects,
when it knows nothing about them? The answer is for the model developer to supply
\emph{data callbacks} to \libadjoint, to give \libadjoint the power to manipulate
these objects as necessary.

This approach entails some extra work on the part of
the model developer; however, this extra work is once-off for each sort of data
structure employed, and can easily be shared between model developers by incorporating
the various sets of data callbacks in \libadjoint itself. Already, \libadjoint includes
data callbacks for interfacing with the PETSc \texttt{Vec} and \texttt{Mat} types (\autoref{sec:petsc_callbacks}),
and contributions in this regard are very welcome.

\newpage
\defapiss{adj_vector}
The \texttt{adj_vector} type is public, i.e.\ it is intended that model developers will read and
write its components. Its definition:
\begin{framed}
\begin{minipage}{\columnwidth}
\begin{ccode}
typedef struct
{
  void* ptr; /* a pointer to the model's data */
  int klass; /* a field to be set by the user 
                in case adj_vector wraps multiple
                separate types */
  int flags; /* for any flags the user might like to set */
} adj_vector;
\end{ccode}
\begin{fortrancode}
  type, bind(c) :: adj_vector
    type(c_ptr) :: ptr ! a pointer to the model's data
    integer(kind=c_int) :: klass ! a field to be set by the user
                                 ! in case adj_vector wraps multiple
                                 ! separate types
    integer(kind=c_int) :: flags ! for any flags the user might like to set
  end type adj_vector
\end{fortrancode}
\end{minipage}
\end{framed}

The \texttt{ptr} attribute takes the address of the model vector to be wrapped; note that
since the \refapi{adj_vector} only has the address of the model vector, the model
vector must not be deallocated before the \refapi{adj_vector} has been used.

The \texttt{klass} attribute exists because it may be desirable to wrap more than
one model datatype as an \refapi{adj_vector}. For example, the Fluidity/ICOM model
\citep{piggott2008} has separate data types \texttt{type(scalar_field)}, \texttt{type(vector_field)}
and \texttt{type(tensor_field)} representing scalar, vector and tensor fields respectively; all
of these can be wrapped by the same \refapi{adj_vector} class, by setting the \texttt{klass} attribute
differently.

The \texttt{flags} attribute stores any flags about the vector that the user wishes to record.
\defapiss{adj_matrix}
Like the \refapi{adj_vector} type, the \texttt{adj_matrix} type is public. As it takes the same
approach to wrapping model matrix types as \refapi{adj_vector} takes to wrapping model vectors,
it has exactly the same definition:
\begin{framed}
\begin{minipage}{\columnwidth}
\begin{ccode}
typedef struct
{
  void* ptr; /* a pointer to the model's data */
  int klass; /* a field to be set by the developer
                in case adj_matrix wraps multiple
                separate types */
  int flags; /* for any flags one might like to set */
} adj_matrix;
\end{ccode}
\begin{fortrancode}
  type, bind(c) :: adj_matrix
    type(c_ptr) :: ptr ! a pointer to the model's data
    integer(kind=c_int) :: klass ! a field to be set by the developer
                                 ! in case adj_matrix wraps multiple
                                 ! separate types
    integer(kind=c_int) :: flags ! for any flags one might like to set
  end type adj_matrix
\end{fortrancode}
\end{minipage}
\end{framed}
Again, the \texttt{klass} attribute exists because it may be desirable to wrap more than
one model datatype as an \refapi{adj_matrix}. For example, the Fluidity/ICOM model
\citep{piggott2008} has separate data types representing CSR, block CSR and matrix-free matrix formats,
all of which can be wrapped by the same \refapi{adj_matrix} class by setting the \texttt{klass} attribute
appropriately.

\subsection{Registering data callbacks}
\defapiss{adj_register_data_callback}
\begin{framed}
\begin{minipage}{\columnwidth}
\begin{ccode}
  int adj_register_data_callback(adj_adjointer* adjointer, int type, 
                                 void (*fnptr)(void));
\end{ccode}
\begin{fortrancode}
  function adj_register_data_callback(adjointer, type, fnptr) result(ierr)
    type(adj_adjointer), intent(inout) :: adjointer
    integer(kind=c_int), intent(in), value :: type
    type(c_funptr), intent(in), value :: fnptr
    integer(kind=c_int) :: ierr
  end function adj_register_data_callback
\end{fortrancode}
\end{minipage}
\end{framed}
This function registers a given data callback with an \refapi{adj_adjointer} so that
\libadjoint can manipulate \refapi{adj_vector} and \refapi{adj_matrix} types. For a list of the
valid values of \texttt{type}, see \autoref{sec:vector_callbacks} and \autoref{sec:matrix_callbacks}.

Example usage in C:
\begin{framed}
\begin{minipage}{\columnwidth}
\begin{ccode}
  ierr = adj_register_data_callback(adjointer, ADJ_VEC_DUPLICATE_CB, 
                                   (void (*)(void)) my_vec_duplicate_proc);
  adj_chkierr(ierr);
\end{ccode}
\end{minipage}
\end{framed}

Example usage in Fortran:
\begin{framed}
\begin{minipage}{\columnwidth}
\begin{fortrancode}
  ierr = adj_register_data_callback(adjointer, ADJ_VEC_DUPLICATE_CB, 
                                    c_funloc(my_vec_duplicate_proc));
  call adj_chkierr(ierr);
\end{fortrancode}
\end{minipage}
\end{framed}

As \libadjoint is only given the function pointer to the data callback, it cannot check that the
interface of the given function is correct. If you register a function which is declared to take 
in an interface different to that which \libadjoint expects, then the model will segfault when
\libadjoint calls that function (or worse, smash the stack). Be very careful when writing data callbacks to use exactly
the interface documented in this manual!

\subsection{Vector callbacks} \label{sec:vector_callbacks}
\defapiss{ADJ_VEC_DUPLICATE_CB}
\begin{framed}
\begin{minipage}{\columnwidth}
\begin{ccode}
  void vec_duplicate(adj_vector x, adj_vector *newx);
\end{ccode}
\begin{fortrancode}
  subroutine vec_duplicate(x, newx) bind(c)
    type(adj_vector), intent(in), value :: x
    type(adj_vector), intent(out) :: newx
  end subroutine vec_duplicate
\end{fortrancode}
\end{minipage}
\end{framed}
This data callback is the fundamental allocation callback for the \refapi{adj_vector} type. \libadjoint
often needs to create new vectors for various tasks; \texttt{x} is an
input model \refapi{adj_vector} to be duplicated, while \texttt{newx} is the new vector
to be allocated. Note that this callback \textbf{must initialise the contents of \texttt{newx} to zero}.

When necessary: always.
\defapiss{ADJ_VEC_AXPY_CB}
\begin{framed}
\begin{minipage}{\columnwidth}
\begin{ccode}
  void vec_axpy(adj_vector *y, adj_scalar alpha, adj_vector x);
\end{ccode}
\begin{fortrancode}
  subroutine vec_axpy(y, alpha, x) bind(c)
    type(adj_vector), intent(inout) :: y
    adj_scalar_f, intent(in), value :: alpha
    type(adj_vector), intent(in), value :: x
  end subroutine vec_axpy
\end{fortrancode}
\end{minipage}
\end{framed}
This data callback adds one vector to another, i.e.\ executes
\begin{equation*}
y \leftarrow y + \alpha x.
\end{equation*}

When necessary: always.
\defapiss{ADJ_VEC_DESTROY_CB}
\begin{framed}
\begin{minipage}{\columnwidth}
\begin{ccode}
  void vec_destroy(adj_vector *x);
\end{ccode}
\begin{fortrancode}
  subroutine vec_destroy(x) bind(c)
    type(adj_vector), intent(inout) :: x
  end subroutine vec_destroy
\end{fortrancode}
\end{minipage}
\end{framed}
This data callback is called when \libadjoint is finished
with a vector; it is used to deallocate its memory and free
its resources.

When necessary: always.
\defapiss{ADJ_VEC_SET_VALUES_CB}
\begin{framed}
\begin{minipage}{\columnwidth}
\begin{ccode}
  void vec_set_values(adj_vector *x, adj_scalar* scalars);
\end{ccode}
\begin{fortrancode}
  subroutine vec_set_values(vec, scalars) bind(c)
    type(adj_vector), intent(inout) :: vec
    adj_scalar_f, dimension(*), intent(in) :: scalars
  end subroutine vec_set_values
\end{fortrancode}
\end{minipage}
\end{framed}
This data callback sets a given \refapi{adj_vector} from an array
of \texttt{adj_scalar}s. The length of the array is the output of
the \refapi{ADJ_VEC_GET_SIZE_CB} data callback.

When necessary: this data callback is necessary when the derivative test flag
is activated on an \refapi{adj_nonlinear_block} with \refapi{adj_nonlinear_block_set_test_derivative}.
\defapiss{ADJ_VEC_GET_SIZE_CB}
\begin{framed}
\begin{minipage}{\columnwidth}
\begin{ccode}
  void vec_get_norm(adj_vector x, int* sz);
\end{ccode}
\begin{fortrancode}
  subroutine vec_get_size(vec, sz) bind(c)
    type(adj_vector), intent(in), value :: vec
    integer(kind=c_int), intent(out) :: sz
  end subroutine vec_get_size
\end{fortrancode}
\end{minipage}
\end{framed}
This data callback returns the number of degrees of freedom in a given \refapi{adj_vector}. The
vector space of all such \refapi{adj_vector}s should be isomorphic to $\mathbb{R}^n$ for some $n$;
this callback returns $n$.

When necessary: this data callback is necessary when the derivative test flag
is activated on an \refapi{adj_nonlinear_block} with \refapi{adj_nonlinear_block_set_test_derivative}.
\defapiss{ADJ_VEC_GET_NORM_CB}
\begin{framed}
\begin{minipage}{\columnwidth}
\begin{ccode}
  void vec_get_norm(adj_vector x, adj_scalar* norm);
\end{ccode}
\begin{fortrancode}
  subroutine vec_norm(x, norm) bind(c)
    type(adj_vector), intent(in), value :: x
    adj_scalar_f, intent(out) :: norm
  end subroutine vec_norm
\end{fortrancode}
\end{minipage}
\end{framed}
This data callback computes the norm of a given vector. It does not matter
which norm is chosen, so long as it satisfies the usual axioms for a norm.

When necessary: this data callback is necessary when the derivative test flag
is activated on an \refapi{adj_nonlinear_block} with \refapi{adj_nonlinear_block_set_test_derivative}.

\defapiss{ADJ_VEC_SET_RANDOM_CB}
\begin{framed}
\begin{minipage}{\columnwidth}
\begin{ccode}
  void vec_set_random(adj_vector *x);
\end{ccode}
\begin{fortrancode}
  subroutine vec_set_random(x) bind(c)
    type(adj_vector), intent(inout) :: x
  end subroutine vec_set_random
\end{fortrancode}
\end{minipage}
\end{framed}
This data callback sets the entries of a given vector \texttt{x} to pseudo-random values.

When necessary: this data callback is necessary when the Hermitian test flag is
activated on an \refapi{adj_block} with \refapi{adj_block_set_test_hermitian},
or on an \refapi{adj_nonlinear_block} with \\\refapi{adj_nonlinear_block_set_test_hermitian}.

\defapiss{ADJ_VEC_DOT_PRODUCT_CB}
\begin{framed}
\begin{minipage}{\columnwidth}
\begin{ccode}
  void vec_dot_product(adj_vector x, adj_vector y, adj_scalar* val);
\end{ccode}
\begin{fortrancode}
  subroutine vec_dot_product(x, y, val) bind(c)
    type(adj_vector), intent(in), value :: x, y
    adj_scalar_f, intent(out) :: val
  end subroutine vec_dot_product
\end{fortrancode}
\end{minipage}
\end{framed}
This data callback computes the dot product (inner product) of two \refapi{adj_vector}s.

When necessary: this data callback is necessary when the Hermitian test flag is
activated on an \refapi{adj_block} with \refapi{adj_block_set_test_hermitian},
or on an \refapi{adj_nonlinear_block} with \\\refapi{adj_nonlinear_block_set_test_hermitian}.

\subsection{Matrix callbacks} \label{sec:matrix_callbacks}
\defapiss{ADJ_MAT_DUPLICATE_CB}
\begin{framed}
\begin{minipage}{\columnwidth}
\begin{ccode}
  void mat_duplicate(adj_matrix X, adj_matrix *newX);
\end{ccode}
\begin{fortrancode}
  subroutine mat_duplicate(X, newX) bind(c)
    type(adj_matrix), intent(in), value :: X
    type(adj_matrix), intent(out) :: newX
  end subroutine mat_duplicate
\end{fortrancode}
\end{minipage}
\end{framed}
This data callback is the fundamental allocation callback for the \refapi{adj_matrix} type. \libadjoint
often needs to create new matrices for various tasks; \texttt{X} is an
input model \refapi{adj_matrix} to be duplicated, while \texttt{newX} is the new matrix
to be allocated. Note that this callback \textbf{must initialise the contents of \texttt{newX} to zero}.

When necessary: always.
\defapiss{ADJ_MAT_AXPY_CB}
\begin{framed}
\begin{minipage}{\columnwidth}
\begin{ccode}
  void mat_axpy(adj_matrix *Y, adj_scalar alpha, adj_matrix X);
\end{ccode}
\begin{fortrancode}
  subroutine mat_axpy(Y, alpha, X) bind(c)
    type(adj_matrix), intent(inout) :: Y
    adj_scalar_f, intent(in), value :: alpha
    type(adj_matrix), intent(in), value :: X
  end subroutine mat_axpy
\end{fortrancode}
\end{minipage}
\end{framed}
This data callback adds one matrix to another, i.e.\ executes
\begin{equation*}
Y \leftarrow Y + \alpha X.
\end{equation*}

When necessary: always.
\defapiss{ADJ_MAT_DESTROY_CB}
\begin{framed}
\begin{minipage}{\columnwidth}
\begin{ccode}
  void mat_destroy(adj_matrix *X);
\end{ccode}
\begin{fortrancode}
  subroutine mat_destroy(X) bind(c)
    type(adj_matrix), intent(inout) :: X
  end subroutine mat_destroy
\end{fortrancode}
\end{minipage}
\end{framed}
This data callback is called when \libadjoint is finished
with a matrix; it is used to deallocate its memory and free
its resources.

When necessary: always.
\subsection{Supplied data callbacks}
If the model uses a common library for its fundamental datatypes, rather than
writing its own, it is possible to distribute the data callbacks with \libadjoint,
so that other model developers do not have to re-write them from scratch. At present, the
only data types with supplied data callbacks are the PETSc \texttt{Vec} and \texttt{Mat} types.
Other data type callbacks for common libraries are very welcome!

\subsection{PETSc data callbacks} \label{sec:petsc_callbacks}
\defapiss{adj_set_petsc_data_callbacks}
\begin{framed}
\begin{minipage}{\columnwidth}
\begin{ccode}
  int adj_set_petsc_data_callbacks(adj_adjointer* adjointer);
\end{ccode}
\begin{fortrancode}
  function adj_set_petsc_data_callbacks(adjointer) result(ierr)
    type(adj_adjointer), intent(inout) :: adjointer
    integer(kind=c_int) :: ierr
  end function adj_set_petsc_data_callbacks
\end{fortrancode}
\end{minipage}
\end{framed}
This function registers the data callbacks for the PETSc \texttt{Vec} and \texttt{Mat} types
supplied with \libadjoint.

\libadjoint also supplies utility functions to convert between \texttt{Vec} and \texttt{Mat}
types and the \texttt{adj_vector} and \texttt{adj_matrix} types that wrap them.
\defapiss{petsc_vec_to_adj_vector}
\begin{framed}
\begin{minipage}{\columnwidth}
\begin{ccode}
  adj_vector petsc_vec_to_adj_vector(Vec* input);
\end{ccode}
\begin{fortrancode}
  function petsc_vec_to_adj_vector(input) result(output)
    Vec, intent(in), target :: input
    type(adj_vector) :: output
  end function petsc_vec_to_adj_vector
\end{fortrancode}
\end{minipage}
\end{framed}
Given a \texttt{Vec}, return an \refapi{adj_vector} that wraps it.
\defapiss{petsc_vec_from_adj_vector}
\begin{framed}
\begin{minipage}{\columnwidth}
\begin{ccode}
  Vec petsc_vec_from_adj_vector(adj_vector input);
\end{ccode}
\begin{fortrancode}
  function petsc_vec_from_adj_vector(input) result(output)
    type(adj_vector), intent(in) :: input
    Vec :: output
  end function petsc_vec_from_adj_vector
\end{fortrancode}
\end{minipage}
\end{framed}
Given an \refapi{adj_vector}, return the \texttt{Vec} that it wraps.
\defapiss{petsc_mat_to_adj_matrix}
\begin{framed}
\begin{minipage}{\columnwidth}
\begin{ccode}
  adj_matrix petsc_mat_to_adj_matrix(Mat* input);
\end{ccode}
\begin{fortrancode}
  function petsc_mat_to_adj_matrix(input) result(output)
    Mat, intent(in), target :: input
    type(adj_matrix) :: output
  end function petsc_mat_to_adj_matrix
\end{fortrancode}
\end{minipage}
\end{framed}
Given a \texttt{Mat}, return an \refapi{adj_matrix} that wraps it.
\defapiss{petsc_mat_from_adj_matrix}
\begin{framed}
\begin{minipage}{\columnwidth}
\begin{ccode}
  Mat petsc_mat_from_adj_matrix(adj_matrix input);
\end{ccode}
\begin{fortrancode}
  function petsc_mat_from_adj_matrix(input) result(output)
    type(adj_matrix), intent(in) :: input
    Mat :: output
  end function petsc_mat_from_adj_matrix
\end{fortrancode}
\end{minipage}
\end{framed}
Given an \refapi{adj_matrix}, return the \texttt{Mat} that it wraps.

\section{Operator callbacks}
\defapis{adj_register_operator_callback}

\section{Source-term callbacks}
\defapis{adj_register_forward_source_callback}
\defapis{adj_register_functional_derivative_callback}

\section{Functional evaluation callbacks}
\defapis{adj_register_functional_callback}
