\chapter{The fundamental idea}

\begin{synopsis}
This chapter attempts to describe the fundamental idea behind \libadjoint, 
and compares and contrasts this idea to the fundamental idea of algorithmic/automatic differentiation (AD).
A high-level overview of the strategy for developing adjoint models with \libadjoint is discussed.
\end{synopsis}
\minitoc
\vspace{\fill}

\index{adjoints!introductory references}
This manual presupposes some passing familiarity with algorithmic differentiation (AD),
and with adjoints. For an introduction to AD, see \citet{rall1996}; for a more mathematical
treatise, see \citet{griewank2003,griewank2008}. For an introduction to adjoints and how
they can be used, see \citet{gunzburger2003,giles2000,errico1997}; for a more mathematical
treatise, see \citet{hinze2009}.

\newpage

\section{The fundamental idea of algorithmic differentiation}
A discrete forward model implements a function that takes in some inputs
and maps those to some outputs. In practice, that function is described
by the developer as a sequence of primitive steps: the function is
a composition of primitive instructions that the programming environment
supplies to the developer, such as addition, multiplication, exponentiation,
and so on.

If we wish to differentiate the model, and the program is the composition
of primitive instructions, then we can simply apply the chain rule:
differentiate each primitive instruction in turn and compose the derivatives
appropriately. The fundamental abstraction of algorithmic differentiation\index{algorithmic differentiation!fundamental idea}\index{automatic differentiation|see{\\algorithmic differentiation}}
is that \emph{a model is a sequence of primitive instructions}.

\section{The fundamental idea of \libadjoint}
\index{libadjoint@\libadjoint!fundamental idea}
The fundamental abstraction of \libadjoint is that \emph{a model is a sequence
of linear solves}. This abstraction shares a similar structure to the fundamental
idea of algorithmic differentiation, but operates at a much higher level:
each linear solve may involve billions of primitive operations.

At its heart, \libadjoint is a library that implements this abstraction. The
forward model is \emph{annotated} with library calls that record the details
of the linear solves performed. This annotation is analogous to the reverse
AD concept of a tape; it records all relevant details of the execution of
the forward model. Each linear solve is annotated to record what is being
computed, what operators are acting on previously computed values, and what
nonlinear dependencies those operators have.

Before discussing the advantages and disadvantages of these two abstractions, 
we should clear up any potential confusion. This abstraction applies equally
well regardless of:
\begin{itemize}
\item whether the forward model uses finite difference, finite
element, or any other discretisation
\item whether the forward model uses explicit or implicit timestepping
\item whether the forward model is nonlinear or not.
\end{itemize}

In the case where explicit timestepping is used, the linear solves
\index{libadjoint@\libadjoint!explicit timestepping}
will have the identity matrix on the left hand side; but they can still
be considered as linear solves. In the nonlinear case, all techniques for
\index{libadjoint@\libadjoint!nonlinear iterations}
solving such systems (such as Newton iteration or Picard iteration) also
boil down to performing a sequence of linear solves for successively better
approximations to the solution.

Note that any computer program may be seen as a sequence of primitive operations, 
whereas the abstraction of \libadjoint is specific to computational models.

\section{Advantages and disadvantages}
\subsection{Algorithmic differentiation}
\index{algorithmic differentiation!advantages}
Algorithmic differentiation is an excellent idea, and it has been enormously
successful. Let us first consider its strengths:
\begin{itemize}
\item In principle, the sequence of primitive instructions is available from
the existing source code of the model, so developer intervention should
be minimal.
\item Algorithmic differentiation is relatively mature: books, commercial
software and consultancy services are available.
\item It works very well: consider the impressive success of projects such as the
MITgcm adjoint \citep{heimbach2002,heimbach2005} and the FLUENT adjoint \citep{bischof2007}.
\end{itemize}

However, algorithmic differentiation is not perfect; there are weaknesses to
the approach:
\index{algorithmic differentiation!disadvantages}
\begin{itemize}
\item An AD tool needs to understand the code in the same manner as a compiler.
However, AD tools typically only support a subset of the language, which rules
out their use on models which exploit such language features.
\item A model is typically a lot more than just the sequence of primitive instructions:
applying AD naively would involve differentiating I/O, linear solvers, dynamic
memory allocation, MPI calls \citep{utke2009}, etc.
\item If the control flow passes through external libraries, then these must also be
differentiated.
\end{itemize}

The fundamental abstraction of AD is a very low-level one, and this means that it is
most powerful when applied to smaller, local pieces of code. If a very large
model is to be differentiated with AD, it typically must be written with this in
mind, in the subset of the language that the AD tool supports, and large amounts
of developer time must be devoted to the continual re-application of the AD tool
as the codebase changes.

\subsection{\libadjoint}

\index{libadjoint@\libadjoint!advantages}
Now let us consider the strengths of the abstraction employed by \libadjoint:
\begin{itemize}
\item It is possible to adjoint models which use language features unsupported 
by AD tools.
\item Implementation details such as the use of external libraries, I/O, dynamic
memory allocation, etc\. are no longer problematic: \libadjoint does not need to
know about them.
\item Whereas an AD tool needs to be re-applied for each change to the model
codebase, the \libadjoint annotation only needs to change much less often,
if the fundamental structure of the equations solved changes.
\item The abstraction makes it possible to offer powerful debugging facilities that enable adjoint models
to be developed much faster than if the adjoint were to be written by hand. % FIXME: reference the chapter later on
\end{itemize}

However, nothing in life comes for free:
\index{libadjoint@\libadjoint!disadvantages}
\begin{itemize}
\item Annotating the model must be done by hand, whereas a sufficiently advanced
AD tool would not require user intervention.
\item The idea is novel, and it remains to be seen if it becomes a popular approach
to developing adjoint models.
\item \libadjoint is under heavy development, although several models have been adjointed
using it.
\end{itemize}

\index{libadjoint@\libadjoint!use of algorithmic differentiation}
The two approaches (AD and \libadjoint) are not competitors; in fact, they complement each
other very well. \libadjoint requires
the developer to supply the derivatives of the nonlinear assembly operators with respect to
their immediate arguments; AD works very well on such small, local pieces of code. By taking
this approach, the pragmatic model developer can use both abstractions where they are strong,
and have the best of both worlds.

\section{The problem \libadjoint solves}
Suppose we have a discrete forward problem
\begin{equation*}
A(u)\cdot u = b(u),
\end{equation*}
where $u$ is the vector of unknown variables, $A(u)$ is the discretised operator,
and $b(u$) is the source term. As we will see in section~\ref{sec:examples}, every discrete model
can written in this form, including time-dependent, nonlinear problems.

Let $J(u)$ be a functional of interest. Then it can be shown \citep{gunzburger2003} that
the discrete adjoint equation is given by
\index{adjoints!discrete adjoint equation}
\begin{equation*}
(A + G - R)^* \lambda = \frac{\partial J}{\partial u},
\end{equation*}
where $\lambda$ is the adjoint solution, $A$ is the original forward operator, $G$ is the derivative of $A$ given by
\begin{equation*}
G = \left({\frac{\partial A}{\partial u}}\right)_{ijk} u_j,
\end{equation*}
$R$ is the Jacobian of the right-hand-side given by
\begin{equation*}
R = \frac{\partial b}{\partial u},
\end{equation*}
and $^*$ means to take the conjugate transpose. If $A$ does not depend on
$u$ (i.e., your operator is linear), then $G=0$. If $b$ does not depend on
$u$, then $R=0$.

For a time-dependent problem, $A(u)$ has a lower triangular triangular structure,
corresponding to the temporal flow of information from the start of the simulation
to the end. By transposition, $(A + G - R)^*$ has an upper triangular structure, where
information propagates from the end of time to the beginning. This is why adjoint
models run backwards in time.

With \libadjoint you describe the block-structure of $A(u)$,
and supply functions which assemble or compute the action of each block in $A$.
With this information, \libadjoint can automatically derive the adjoint equations
and assemble each equation in turn. \libadjoint also keeps track of when each
variable of the forward and adjoint equations are needed, and can automatically
forget them when nothing further depends on them. This frees the developer from
complicated lifecycle management, and greatly simplifies the process of adjoint model
development.

\section{The structure of \libadjoint}
The fundamental object of \libadjoint is an \refapi{adj_adjointer}. The forward model is annotated
to record the sequence of linear solves that the model has executed (\refapi{adj_register_equation}); the \refapi{adj_adjointer} is
the object that holds the record. When the \refapi{adj_adjointer} has this information, it can derive
the structure of each adjoint equation: what should be assembled or differentiated for each linear
solve in the adjoint run. If the model developer then
supplies callbacks to the \refapi{adj_adjointer} (\refapi{adj_register_operator_callback}) to allow \libadjoint to assemble each
operator that appears in the forward equation, \libadjoint is
able to assemble each adjoint equation as it is to be solved (\refapi{adj_get_adjoint_equation}).

\libadjoint is organised in a data-structure-neutral manner, inspired by the design
of Zoltan \citep{devine2002}. \libadjoint frequently needs to manipulate vectors
and matrices, but the objects that represent these (\refapi{adj_vector} and \refapi{adj_matrix})
are wrappers around user data types that are opaque to \libadjoint. \libadjoint manipulates
these opaque data types through data callbacks supplied by the developer. Currently, \libadjoint
\index{PETSc!data types}
comes with data callbacks to interoperate with the PETSc \texttt{Vec} and \texttt{Mat} types \citep{balay2010}.
If your model uses other data types, then a handful of callbacks must be supplied to \libadjoint (\refapi{adj_register_data_callback}).
If these types are not specific to your model, then please consider contributing them back!

Great care has been taken to make the API of \libadjoint as consistent as possible. All functions
return an error code. \refapi{ADJ_OK} means that the function worked as normal. A negative
return code means that a warning was issued; a positive return code means that an error occurred.
Every call to \libadjoint should be followed by \refapi{adj_chkierr}, which inspects the return
code and prints out a helpful error message.

\libadjoint is written in C, and currently offers bindings for C, Fortran 90 and Python. The Fortran 90 bindings
are implemented using the \texttt{iso_c_binding} facility \citep{metcalf2004}, and therefore require a modern
Fortran compiler. This facility is supported by the latest releases of all major Fortran compilers; in particular,
if you are developing using \texttt{gfortran}, you must use version 4.5 or greater.
Of course, bindings to other languages are very welcome, especially C++.

\section{The process of adjoint model development with \libadjoint}
One of the advantages of using \libadjoint to develop an adjoint model is that the process
is very systematic and incremental. 

Firstly, if the core data types of your model are unknown to \libadjoint, then you
should write the data callbacks so that \libadjoint can allocate, destroy, and add
vectors and matrices.

Secondly, annotate the first equation in your model, and write the operator
callbacks for the matrices that appear. Here, you should start to
exploit the powerful debugging and development facilities in \libadjoint. The
annotation builds a ``tape'' of the model execution, and \libadjoint uses this
to rewind the model. But \libadjoint can use this tape to replay the forward
model back again; furthermore, \libadjoint can automatically compare the output
of the replay against the values of your original forward model. Therefore,
annotate the first equation, and then check the consistency of the tape with the
forward model; once you are happy with the first equation, iterate until you have
successfully annotated all equations, and can replay the model run in its
entirety.

When the forward replay works correctly, the adjoint model is almost ready.
There are two other major sources of developer error: inconsistency of
transposed operators, and incorrectness of derivative code. \libadjoint
offers debugging tools to eliminate both classes of error. 

In order to 
check the correctness of operator transposition, we employ the identity
that
\begin{equation*}
\langle y, Ax \rangle = \langle A^* y, x \rangle
\end{equation*}
for any suitable vectors $x, y$ and an operator $A$. \libadjoint computes both
sides of the identity and checks that they are equal to machine precision. This
check is particularly useful for operators which are never explicitly
represented as matrices.

In order to assemble the adjoint equation, \libadjoint also requires
callbacks which compute the derivative of the nonlinear assembly operators
with respect to their immediate inputs. (As we will see later, these
derivative routines can be generated easily by applying algorithmic
differentiation.) In order to check the consistency of any derivative,\index{algorithmic differentiation!derivative test}
we employ the identity that
\begin{equation*}
\left|\left| f(x + \delta x) - f(x) \right|\right|
\end{equation*}
converges to zero at $O(\left|\left| \delta x \right|\right|)$ as $\left|\left|\delta x\right|\right| \rightarrow 0$, whereas
\begin{equation*}
\left|\left| f(x + \delta x) - f(x) - \nabla f(x)\cdot \delta x\right|\right|
\end{equation*}
converges at $O(\left|\left| \delta x \right|\right|^2)$. This check is
particularly useful when the nonlinear assembly routine had to be
modified so that an algorithmic differentiation tool could be applied,
as it asserts the correspondence between the original model code
and the derivative of the modified model code.

Once the forward replay works correctly, and these checks have been
passed, then the assembled adjoint equations are almost certainly correct.

