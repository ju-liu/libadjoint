\chapter{The fundamental idea}

\begin{synopsis}
This chapter attempts to describe the fundamental idea behind \libadjoint, 
and compares and contrasts this idea to the fundamental idea of algorithmic/automatic differentiation (AD).
A high-level overview of the strategy for developing adjoint models with \libadjoint is discussed.
\end{synopsis}
\vspace{\fill}
\minitoc

This manual presupposes some passing familiarity with algorithmic differentiation (AD),
and with adjoints. For an introduction to AD, see \citet{rall1996}; for a more mathematical
treatise, see \citet{griewank2003,griewank2008}. For an introduction to adjoints and how
they can be used, see \citet{gunzburger2003,giles2000,errico1997}; for a more mathematical
treatise, see \citet{hinze2009}.

\newpage

\section{The fundamental idea of algorithmic differentiation}
A discrete forward model implements a function that takes in some inputs
and maps those to some outputs. In practice, that function is described
by the programmer as a sequence of primitive steps: the function is
a composition of primitive instructions that the programming environment
supplies to the programmer, such as addition, multiplication, exponentiation,
and so on.

If we wish to differentiate the model, and the program is the composition
of primitive instructions, then we can simply apply the chain rule:
differentiate each primitive instruction in turn and compose the derivatives
appropriately. The fundamental abstraction of algorithmic differentiation
is that \emph{a model is a sequence of primitive instructions}.

\section{The fundamental idea of \libadjoint}
The fundamental abstraction of \libadjoint is that \emph{a model is a sequence
of linear solves}. This abstraction shares a similar structure to the fundamental
idea of algorithmic differentiation, but operates at a much higher level:
each linear solve may involve billions of primitive operations.

Before discussing the advantages and disadvantages of these two abstractions, 
we should clear up any potential confusion. This abstraction applies equally
well regardless of:
\begin{itemize}
\item whether the forward model uses finite difference, finite
element, or any other discretisation
\item whether the forward model uses explicit or implicit timestepping
\item whether the forward model is nonlinear or not.
\end{itemize}

In the case where explicit timestepping is used, the linear solves
will have the identity matrix on the left hand side; but they can still
be considered as linear solves. In the nonlinear case, all techniques for
solving such systems (such as Newton iteration or Picard iteration) also
boil down to performing a sequence of linear solves for successively better
approximations to the solution.

Note that any computer program may be seen as a sequence of primitive operations, 
whereas the abstraction of \libadjoint is specific to computational models.

\section{Advantages and disadvantages}
\section{A strategy for adjoint model development with \libadjoint}
