\chapter{The fundamental idea}

\begin{synopsis}
This chapter attempts to describe the fundamental idea behind \libadjoint, 
and compares and contrasts this idea to the fundamental idea of algorithmic/automatic differentiation (AD).
A high-level overview of the strategy for developing adjoint models with \libadjoint is discussed.
\end{synopsis}
\vspace{\fill}
\minitoc

This manual presupposes some passing familiarity with algorithmic differentiation (AD),
and with adjoints. For an introduction to AD, see \citet{rall1996}; for a more mathematical
treatise, see \citet{griewank2003,griewank2008}. For an introduction to adjoints and how
they can be used, see \citet{gunzburger2003,giles2000,errico1997}; for a more mathematical
treatise, see \citet{hinze2009}.

\newpage

\section{The fundamental idea of algorithmic differentiation}
A discrete forward model implements a function that takes in some inputs
and maps those to some outputs. In practice, that function is described
by the developer as a sequence of primitive steps: the function is
a composition of primitive instructions that the programming environment
supplies to the developer, such as addition, multiplication, exponentiation,
and so on.

If we wish to differentiate the model, and the program is the composition
of primitive instructions, then we can simply apply the chain rule:
differentiate each primitive instruction in turn and compose the derivatives
appropriately. The fundamental abstraction of algorithmic differentiation
is that \emph{a model is a sequence of primitive instructions}.

\section{The fundamental idea of \libadjoint}
The fundamental abstraction of \libadjoint is that \emph{a model is a sequence
of linear solves}. This abstraction shares a similar structure to the fundamental
idea of algorithmic differentiation, but operates at a much higher level:
each linear solve may involve billions of primitive operations.

At its heart, \libadjoint is a library that implements this abstraction. The
forward model is \emph{annotated} with library calls that record the details
of the linear solves performed. This annotation is analogous to the reverse
AD concept of a tape; it records all relevant details of the execution of
the forward model. Each linear solve is annotated to record what is being
computed, what operators are acting on previously computed values, and what
nonlinear dependencies those operators have.

Before discussing the advantages and disadvantages of these two abstractions, 
we should clear up any potential confusion. This abstraction applies equally
well regardless of:
\begin{itemize}
\item whether the forward model uses finite difference, finite
element, or any other discretisation
\item whether the forward model uses explicit or implicit timestepping
\item whether the forward model is nonlinear or not.
\end{itemize}

In the case where explicit timestepping is used, the linear solves
will have the identity matrix on the left hand side; but they can still
be considered as linear solves. In the nonlinear case, all techniques for
solving such systems (such as Newton iteration or Picard iteration) also
boil down to performing a sequence of linear solves for successively better
approximations to the solution.

Note that any computer program may be seen as a sequence of primitive operations, 
whereas the abstraction of \libadjoint is specific to computational models.

\section{Advantages and disadvantages}
\subsection{Algorithmic differentiation}
Algorithmic differentiation is an excellent idea, and it has been enormously
successful. Let us first consider its strengths:
\begin{itemize}
\item In principle, the sequence of primitive instructions is available from
the existing source code of the model, so developer intervention should
be minimal.
\item Algorithmic differentiation is relatively mature: books, commercial
software and consultancy services are available.
\item It works very well: consider the impressive success of projects such as the
MITgcm adjoint \citep{heimbach2002,heimbach2005} and the FLUENT adjoint \citep{bischof2007}.
\end{itemize}

However, algorithmic differentiation is not perfect; there are weaknesses to
the approach:
\begin{itemize}
\item An AD tool needs to understand the code in the same manner as a compiler.
However, AD tools typically only support a subset of the language, which rules
out their use on models which exploit such language features.
\item A model is typically a lot more than just the sequence of primitive instructions:
applying AD naively would involve differentiating I/O, linear solvers, dynamic
memory allocation, MPI calls \citep{utke2009}, etc.
\item If the control flow passes through external libraries, then these must also be
differentiated.
\end{itemize}

The fundamental abstraction of AD is a very low-level one, and this means that it is
most powerful when applied to smaller, local pieces of code. If a very large
model is to be differentiated with AD, it typically must be written with this in
mind, in the subset of the language that the AD tool supports, and large amounts
of developer time must be devoted to the continual re-application of the AD tool
as the codebase changes.

\subsection{\libadjoint}

Now let us consider the strengths of the abstraction employed by \libadjoint:
\begin{itemize}
\item It is possible to adjoint models which use language features unsupported 
by AD tools.
\item Implementation details such as the use of external libraries, I/O, dynamic
memory allocation, etc. are no longer problematic: \libadjoint does not need to
know about them.
\item Whereas an AD tool needs to be re-applied for each change to the model
codebase, the \libadjoint annotation only needs to change much less often,
if the fundamental structure of the equations solved changes.
\item The abstraction makes it possible to offer powerful debugging facilities that enable adjoint models
to be developed much faster than if the adjoint were to be written by hand. % FIXME: reference the chapter later on
\end{itemize}

However, nothing in life comes for free:
\begin{itemize}
\item Annotating the model must be done by hand, whereas a sufficiently advanced
AD tool would not require user intervention.
\item The idea is novel, and it remains to be seen if it becomes a popular approach
to developing adjoint models.
\item \libadjoint is under heavy development, although several models have been adjointed
using it.
\end{itemize}

The two approaches (AD and \libadjoint) are not competitors; in fact, they complement each
other very well. \libadjoint requires
the developer to supply the derivatives of the nonlinear assembly operators with respect to
their immediate arguments; AD works very well on such small, local pieces of code. By taking
this approach, the pragmatic model developer can use both abstractions where they are strong,
and have the best of both worlds.

\section{A strategy for adjoint model development with \libadjoint}
