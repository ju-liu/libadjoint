\chapter{Datatypes}
\begin{synopsis}
\end{synopsis}
\minitoc
\vspace{\fill}
\newpage

\defapi{adj_adjointer}
\defapis{adj_create_adjointer}
\begin{boxwithtitle}{\texttt{adj_create_adjointer}}
\begin{minipage}{\columnwidth}
\begin{ccode}
  int adj_create_adjointer(adj_adjointer* adjointer);
\end{ccode}
\begin{fortrancode}
  function adj_create_adjointer(adjointer) result(ierr) 
    type(adj_adjointer), intent(inout) :: adjointer
    integer(kind=c_int) :: ierr
  end function adj_create_adjointer
\end{fortrancode}
\end{minipage}
\end{boxwithtitle}

This function creates an \refapi{adj_adjointer} object. 
It stores the required details for the assembly of the adjoint equations.
In particular the annotation of the forward model and all associated callbacks routines.


\defapis{adj_destroy_adjointer}
\begin{boxwithtitle}{\texttt{adj_destroy_adjointer}}
\begin{minipage}{\columnwidth}
\begin{ccode}
  int adj_destroy_adjointer(adj_adjointer* adjointer);
\end{ccode}
\begin{fortrancode}
  function adj_destroy_adjointer(adjointer) result(ierr) 
    type(adj_adjointer), intent(inout) :: adjointer
    integer(kind=c_int) :: ierr
  end function adj_destroy_adjointer
\end{fortrancode}
\end{minipage}
\end{boxwithtitle}

This function destroys the supplied \refapi{adj_adjointer} object.


\defapis{adj_deactivate_adjointer}
\begin{boxwithtitle}{\texttt{adj_deactivate_adjointer}}
\begin{minipage}{\columnwidth}
\begin{ccode}
  int adj_deactivate_adjointer(adj_adjointer* adjointer);
\end{ccode}
\begin{fortrancode}
  function adj_deactivate_adjointer(adjointer) result(ierr) 
    type(adj_adjointer), intent(inout) :: adjointer
    integer(kind=c_int) :: ierr
  end function adj_deactivate_adjointer
\end{fortrancode}
\end{minipage}
\end{boxwithtitle}

This function deactivates the supplied \refapi{adj_adjointer}.
In particular, the model annotation, the callback registration and the recording functions return \refapi{ADJ_OK} immediately. 

This functionality is primarily used to deactivate the adjoint features for cases where only the forward model result is of interest, 
but compiling out the calls to \libadjoint is impractical.


\defapi{adj_variable}

\defapis{adj_create_variable}
\begin{boxwithtitle}{\texttt{adj_create_variable}}
\begin{minipage}{\columnwidth}
\begin{ccode}
  int adj_create_variable(char* name, int timestep, int iteration, int auxiliary, 
                          adj_variable* var);
\end{ccode}
\begin{fortrancode}
  function adj_create_variable(name, timestep, iteration, auxiliary, variable) 
           result(ierr)
    character(len=*), intent(in) :: name
    integer, intent(in) :: timestep, iteration
    logical, intent(in) :: auxiliary
    type(adj_variable), intent(out) :: variable
    integer :: ierr
  end function adj_create_variable
\end{fortrancode}
\end{minipage}
\end{boxwithtitle}

This function creates an \texttt{adj_variable} with name \texttt{name}, timestep \texttt{timestep} and iteration \texttt{iteration}.
The \texttt{auxiliary} option defines the type of the variable and must be either \texttt{ADJ_NORMAL_VARIABLE} or \texttt{ADJ_AUXILIARY_VARIABLE}.

Auxiliary variables are variables for which no equation is solved, but are necessary for the evaluation of an operator.
A typical example for an auxiliary variable is the computational mesh, which is needed for the assembly of the discrete operator but for which no equation is solved during the simulation.
An alternative way of passing data objects to callback functions is by using the \texttt{context} pointer in \refapi{adj_create_block} and \refapi{adj_create_nonlinear_block}.

Helper functions are available to change and query the properties of a variable after initialisation, see below. 



\defapis{adj_variable_get_name}
\begin{boxwithtitle}{\texttt{adj_variable_get_name}}
\begin{minipage}{\columnwidth}
\begin{ccode}
  int adj_variable_get_name(adj_variable var, char** name);
\end{ccode}
\begin{fortrancode}
  function adj_variable_get_name(variable, name) result(ierr)
    type(adj_variable), intent(in) :: variable
    character(len=*), intent(out) :: name
    integer :: ierr
  end function adj_variable_get_name
\end{fortrancode}
\end{minipage}
\end{boxwithtitle}

This function queries the \texttt{name} of an \texttt{adj_variable}.

\defapis{adj_variable_get_timestep}
\begin{boxwithtitle}{\texttt{adj_variable_get_timestep}}
\begin{minipage}{\columnwidth}
\begin{ccode}
  int adj_variable_get_timestep(adj_variable var, int* timestep);
\end{ccode}
\begin{fortrancode}
  function adj_variable_get_timestep(var, timestep) result(ierr)
    type(adj_variable), intent(in), value :: var
    integer(kind=c_int), intent(out) :: timestep
    integer(kind=c_int) :: ierr
  end function adj_variable_get_timestep
\end{fortrancode}
\end{minipage}
\end{boxwithtitle}

This function queries the \texttt{timestep} of an \texttt{adj_variable}.


\defapis{adj_variable_get_iteration}
\begin{boxwithtitle}{\texttt{adj_variable_get_iteration}}
\begin{minipage}{\columnwidth}
\begin{ccode}
  int adj_variable_get_iteration(adj_variable var, int* iteration);
\end{ccode}
\begin{fortrancode}
  function adj_variable_get_iteration(var, iteration) result(ierr) 
    type(adj_variable), intent(in), value :: var
    integer(kind=c_int), intent(out) :: iteration
    integer(kind=c_int) :: ierr
  end function adj_variable_get_iteration
\end{fortrancode}
\end{minipage}
\end{boxwithtitle}

This function queries the \texttt{iteration} of an \texttt{adj_variable}.


\defapis{adj_variable_set_auxiliary}
\begin{boxwithtitle}{\texttt{adj_variable_set_auxiliary}}
\begin{minipage}{\columnwidth}
\begin{ccode}
  int adj_variable_set_auxiliary(adj_variable* var, int auxiliary);
\end{ccode}
\begin{fortrancode}
  function adj_variable_set_auxiliary(var, auxiliary) result(ierr)
    type(adj_variable), intent(inout) :: var
    logical, intent(in) :: auxiliary
    integer(kind=c_int) :: ierr
  end function adj_variable_set_auxiliary
\end{fortrancode}
\end{minipage}
\end{boxwithtitle}

This function is used to define the type of an \texttt{adj_variable}.
\texttt{auxiliary} must be either \texttt{ADJ_NORMAL_VARIABLE} \index{ADJ_NORMAL_VARIABLE} or \texttt{ADJ_AUXILIARY_VARIABLE}\index{ADJ_AUXILIARY_VARIABLE}.

\defapi{adj_block}

\defapis{adj_create_block}
\begin{boxwithtitle}{\texttt{adj_create_block}}
\begin{minipage}{\columnwidth}
\begin{ccode}
  int adj_create_block(char* name, adj_nonlinear_block* nblock, void* context, 
                       adj_block* block);
\end{ccode}
\begin{fortrancode}
  function adj_create_block(name, nblock, context, block) result(ierr)
    character(len=*), intent(in) :: name
    type(adj_nonlinear_block), intent(in), optional, target :: nblock
    type(c_ptr), intent(in), optional :: context
    type(adj_block), intent(inout) :: block
    integer :: ierr
  end function adj_create_block
\end{fortrancode}
\end{minipage}
\end{boxwithtitle}

This function creates an \texttt{adj_block} with the name \texttt{name}.
If the block is nonlinear, i.e. its entries depend on the solution, the corresponding \refapi{adj_nonlinear_block} is passed as \texttt{nblock}. 
Otherwise, a \texttt{NULL} pointer is to be passed in C and the argument can be ignored in Fortran.
The void pointer \texttt{context} allows the developer to pass arbitrary data to the block's callbacks (\refapi{ADJ_BLOCK_ACTION_CB}, \refapi{ADJ_BLOCK_ASSEMBLY_CB}). 
The result is stored in \texttt{block}. 

In addition, \texttt{adj_block} has properties that are not directly accessible during the initialisation:

\begin{description}
  \item[hermitian] A flag indicating if the hermitian of the operator is to be used. Default: \texttt{ADJ_FALSE}
  \item[coefficient] A scaling factor. Default: 1.0
\end{description}

Helper functions are available to change these values, see below. 


\defapis{adj_block_set_coefficient}
\begin{boxwithtitle}{\texttt{adj_block_set_coefficient}}
\begin{minipage}{\columnwidth}
\begin{ccode}
  int adj_block_set_coefficient(adj_block* block, adj_scalar coefficient);
\end{ccode}
\begin{fortrancode}
  function adj_nonlinear_block_set_coefficient(nblock, coefficient) result(ierr)
    type(adj_nonlinear_block), intent(inout) :: nblock
    adj_scalar_f, intent(in), value :: coefficient
    integer(kind=c_int) :: ierr
  end function adj_nonlinear_block_set_coefficient
\end{fortrancode}
\end{minipage}
\end{boxwithtitle}

This function sets the scaling factor of the supplied block to \texttt{coefficient}. 


\defapis{adj_block_set_hermitian}
\begin{boxwithtitle}{\texttt{adj_block_set_hermitian}}
\begin{minipage}{\columnwidth}
\begin{ccode}
  int adj_block_set_hermitian(adj_block* block, int hermitian);
\end{ccode}
\begin{fortrancode}
  function adj_block_set_hermitian(block, hermitian) result(ierr)
    type(adj_block), intent(inout) :: block
    logical, intent(in) :: hermitian
    integer(kind=c_int) :: ierr
  end function adj_block_set_hermitian
\end{fortrancode}
\end{minipage}
\end{boxwithtitle}

This function sets the hermitian flag of the supplied \texttt{block}. \texttt{hermitian} must be \texttt{ADJ_TRUE} or \texttt{ADJ_FALSE}.

\defapis{adj_destroy_block}
\begin{boxwithtitle}{\texttt{adj_destroy_block}}
\begin{minipage}{\columnwidth}
\begin{ccode}
  int adj_destroy_block(adj_block* block);
\end{ccode}
\begin{fortrancode}
  function adj_destroy_block(block) result(ierr) 
    type(adj_block), intent(inout) :: block
    integer(kind=c_int) :: ierr
  end function adj_destroy_block
\end{fortrancode}
\end{minipage}
\end{boxwithtitle}

Destroys the block \texttt{block}.


\defapi{adj_nonlinear_block}

\defapis{adj_create_nonlinear_block}
\begin{boxwithtitle}{\texttt{adj_create_nonlinear_block}}
\begin{minipage}{\columnwidth}
\begin{ccode}
  int adj_create_nonlinear_block(char* name, int ndepends, adj_variable* depends, 
                                 void* context, adj_nonlinear_block* nblock);
\end{ccode}
\begin{fortrancode}
  function adj_create_nonlinear_block(name, depends, coefficient, context, nblock) 
           result(ierr)
    character(len=*), intent(in) :: name
    type(adj_variable), intent(in), dimension(:) :: depends
    adj_scalar_f, intent(in), optional :: coefficient
    type(c_ptr), intent(in), optional :: context
    type(adj_nonlinear_block), intent(out) :: nblock
    integer :: ierr
  end function adj_create_nonlinear_block
\end{fortrancode}
\end{minipage}
\end{boxwithtitle}

This function creates an \texttt{adj_nonlinear_block} with name \texttt{name}.
The dependency variables \texttt{depends} are passed as an array of \texttt{adj_variable}s of length \texttt{ndepends}.
At least one dependency must be given.
The void pointer \texttt{context} allows the developer to pass arbitrary data to the nonlinear block's callbacks (\refapi{ADJ_NBLOCK_DERIVATIVE_ACTION_CB}, \refapi{ADJ_NBLOCK_ACTION_CB}). 

The resulting nonlinear block is stored in \texttt{nblock}. 

In addition, \texttt{adj_nonlinear_block} has a scaling property that is not directly accessible during the initialisation. 
Its default value is set to 1.0, but can be changed using \refapi{adj_nonlinear_block_set_coefficient}, see below.


\defapis{adj_nonlinear_block_set_coefficient}
\begin{boxwithtitle}{\texttt{adj_nonlinear_block_set_coefficient}}
\begin{minipage}{\columnwidth}
\begin{ccode}
  int adj_nonlinear_block_set_coefficient(adj_nonlinear_block* nblock, 
                                          adj_scalar coefficient);
\end{ccode}
\begin{fortrancode}
  function adj_nonlinear_block_set_coefficient(nblock, coefficient) result(ierr) 
    type(adj_nonlinear_block), intent(inout) :: nblock
    adj_scalar_f, intent(in), value :: coefficient
    integer(kind=c_int) :: ierr
  end function adj_nonlinear_block_set_coefficient
\end{fortrancode}
\end{minipage}
\end{boxwithtitle}

This function sets the scaling factor of the supplied nonlinear block to \texttt{coefficient}. 


\defapis{adj_destroy_nonlinear_block}
\begin{boxwithtitle}{\texttt{adj_destroy_nonlinear_block}}
\begin{minipage}{\columnwidth}
\begin{ccode}
  int adj_destroy_nonlinear_block(adj_nonlinear_block* nblock);
\end{ccode}
\begin{fortrancode}
  function adj_destroy_nonlinear_block(nblock) result(ierr)
    type(adj_nonlinear_block), intent(inout) :: nblock
    integer(kind=c_int) :: ierr
  end function adj_destroy_nonlinear_block
\end{fortrancode}
\end{minipage}
\end{boxwithtitle}

Destroys the nonlinear block \texttt{nblock}.

\defapi{adj_equation}

\defapis{adj_create_equation}
\begin{boxwithtitle}{\texttt{adj_create_equation}}
\begin{minipage}{\columnwidth}
\begin{ccode}
  int adj_create_equation(adj_variable var, int nblocks, adj_block* blocks, 
                          adj_variable* targets, adj_equation* equation);
\end{ccode}
\begin{fortrancode}
  function adj_create_equation(variable, blocks, targets, equation) result(ierr)
    type(adj_variable), intent(in), value :: variable
    type(adj_block), dimension(:), intent(in) :: blocks
    type(adj_variable), dimension(:), intent(in) :: targets
    type(adj_equation), intent(inout) :: equation
    integer :: ierr
  end function adj_create_equation
\end{fortrancode}
\end{minipage}
\end{boxwithtitle}

This function creates an \refapi{adj_equation}.

The arguments \texttt{blocks} and \texttt{targets} are arrays of length \texttt{nblocks} that describe the equation:
The first block of the blocks array is to be multiplied with the first variable in the targets array, 
the second pairs with the second, and so on.
\texttt{var} is the variable to be solved for in this equation.
Hence, the variable to be solved for, \texttt{var}, must exist in \texttt{targets}. If that is not the case an \refapi{ADJ_ERR_INVALID_INPUTS} is returned.


\defapis{adj_equation_set_rhs_dependencies}
\begin{boxwithtitle}{\texttt{adj_equation_set_rhs_dependencies}}
\begin{minipage}{\columnwidth}
\begin{ccode}
  int adj_equation_set_rhs_dependencies(adj_equation* equation, int nrhsdeps, 
                                        adj_variable* rhsdeps, void* context);
\end{ccode}
\begin{fortrancode}
  function adj_equation_set_rhs_dependencies(equation, rhsdeps, context) 
           result(ierr)
    type(adj_equation), intent(inout) :: equation
    type(adj_variable), dimension(:), intent(in), optional :: rhsdeps
    type(c_ptr), intent(in), value, optional :: context
    integer(kind=c_int) :: ierr
  end function adj_equation_set_rhs_dependencies
\end{fortrancode}
\end{minipage}
\end{boxwithtitle}

Sets the dependency variables and the context for the right hand side evaluation of \texttt{equation}. 

The array \texttt{rhsdeps} is an array of length \texttt{nrhsdeps} containing the dependency variables.
The void pointer \texttt{context} allows the developer to pass arbitrary data to the forward source callbacks. 

This information will passed to the source term callback function, if a source term is defined (see \refapi{adj_register_forward_source_callback}).
If the right hand side is not needed or the source term has no dependencies, this function can be ignored.


\defapis{adj_register_equation}
\begin{boxwithtitle}{\texttt{adj_register_equation}}
\begin{minipage}{\columnwidth}
\begin{ccode}
  int adj_register_equation(adj_adjointer* adjointer, adj_equation equation);
\end{ccode}
\begin{fortrancode}
  function adj_register_equation(adjointer, equation) result(ierr)
    type(adj_adjointer), intent(inout) :: adjointer
    type(adj_equation), intent(in), value :: equation
    integer(kind=c_int) :: ierr
  end function adj_register_equation
\end{fortrancode}
\end{minipage}
\end{boxwithtitle}

Registers the equation \texttt{equation} to the \texttt{adjointer}.



\defapis{adj_destroy_equation}
\begin{boxwithtitle}{\texttt{adj_destroy_equation}}
\begin{minipage}{\columnwidth}
\begin{ccode}
  int adj_destroy_equation(adj_equation* equation);
\end{ccode}
\begin{fortrancode}
  function adj_destroy_equation(equation) result(ierr)
    type(adj_equation), intent(inout) :: equation
    integer(kind=c_int) :: ierr
  end function adj_destroy_equation
\end{fortrancode}
\end{minipage}
\end{boxwithtitle}

Destroys the equation \texttt{equation}.


\defapi{adj_storage_data}

\defapis{adj_storage_memory_copy}
\begin{boxwithtitle}{\texttt{adj_storage_memory_copy}}
\begin{minipage}{\columnwidth}
\begin{ccode}
  int adj_storage_memory_copy(adj_vector val, adj_storage_data* data);
\end{ccode}
\begin{fortrancode}
  function adj_storage_memory_copy(val, mem) result(ierr)
    type(adj_vector), intent(in), value :: val
    type(adj_storage_data), intent(inout) :: mem
    integer(kind=c_int) :: ierr
  end function adj_storage_memory_copy
\end{fortrancode}
\end{minipage}
\end{boxwithtitle}

Creates an \refapi{adj_storage_data} object that contains the supplied \refapi{adj_vector} \texttt{val}.

When the resulting \refapi{adj_storage_data} is used to record a variable (see \refapi{adj_record_variable}), 
a copy of \texttt{val} is created using the \refapi{ADJ_VEC_DUPLICATE_CB} and \refapi{ADJ_VEC_AXPY_CB} data callbacks.
Therefore, \texttt{val} can be destroyed in the model after the variable has been recorded. 


\defapis{adj_storage_memory_incref}
\begin{boxwithtitle}{\texttt{adj_storage_memory_incref}}
\begin{minipage}{\columnwidth}
\begin{ccode}
  int adj_storage_memory_incref(adj_vector val, adj_storage_data* data);
\end{ccode}
\begin{fortrancode}
  function adj_storage_memory_incref(val, mem) result(ierr)
    type(adj_vector), intent(in), value :: val
    type(adj_storage_data), intent(inout) :: mem
    integer(kind=c_int) :: ierr
  end function adj_storage_memory_incref
\end{fortrancode}
\end{minipage}
\end{boxwithtitle}

Creates an \refapi{adj_storage_data} object that contains the supplied \refapi{adj_vector} \texttt{val}.

When the resulting \refapi{adj_storage_data} is used to record a variable (see \refapi{adj_record_variable}), 
then only the data pointer to \texttt{var} is recorded.
Therefore the model developer has to ensure that the underlying data remains valid.
This strategy is usually used in combination with reference counting.


\defapis{adj_record_variable}
\begin{boxwithtitle}{\texttt{adj_record_variable}}
\begin{minipage}{\columnwidth}
\begin{ccode}
  int adj_record_variable(adj_adjointer* adjointer, adj_variable var, 
                          adj_storage_data storage);
\end{ccode}
\begin{fortrancode}
  function adj_record_variable(adjointer, variable, storage) result(ierr) 
    type(adj_adjointer), intent(inout) :: adjointer
    type(adj_variable), intent(in), value :: variable
    type(adj_storage_data), intent(in), value :: storage
    integer(kind=c_int) :: ierr
  end function adj_record_variable
\end{fortrancode}
\end{minipage}
\end{boxwithtitle}

Records the variable \texttt{var} in \texttt{adjointer} with the data and storage strategy defined in \texttt{storage}.


\defapi{adj_dict}

During the adjoint model development it is very common that the developer wants to store additional information about \libadjoint data objects 
and access these information later on in the codebase.
\refapi{adj_dict} is an implementation of a dictionary class for strings that can be used for this purpose.

A typical example are solver options that one might want to assign to every \texttt{adj_variable} during the annotation of the forward model.
In the adjoint main loop, these options can then be reused to solve the adjoint equations accordingly.
This can be implemented with an \refapi{adj_dict} that maps the name of the \refapi{adj_variable} to a string with the solver options.

\defapis{adj_dict_init}
\begin{boxwithtitle}{\texttt{adj_dict_init}}
\begin{minipage}{\columnwidth}
\begin{ccode}
  int adj_dict_init(adj_dictionary* dict);
\end{ccode}
\begin{fortrancode}
  function adj_dict_init(dict) result(ierr)
    type(adj_dictionary), intent(inout) :: dict
    integer(kind=c_int) :: ierr
  end function adj_dict_init
\end{fortrancode}
\end{minipage}
\end{boxwithtitle}

Creates a dictionary \texttt{dict}.


\defapis{adj_dict_set}
\begin{boxwithtitle}{\texttt{adj_dict_set}}
\begin{minipage}{\columnwidth}
\begin{ccode}
  int adj_dict_set(adj_dictionary* dict, char* key, char* value);
\end{ccode}
\begin{fortrancode}
  function adj_dict_set(dict, key, value) result(ierr)
    type(adj_dictionary), intent(inout) :: dict
    character(len=*), intent(in) :: key, value
    integer :: ierr
  end function adj_dict_set
\end{fortrancode}
\end{minipage}
\end{boxwithtitle}

Creates or updates an entry in the dictionary \texttt{dict} with key \texttt{key} and value \texttt{value}.


\defapis{adj_dict_find}
\begin{boxwithtitle}{\texttt{adj_dict_find}}
\begin{minipage}{\columnwidth}
\begin{ccode}
  int adj_dict_find(adj_dictionary* dict, char* key, char** value);
\end{ccode}
\begin{fortrancode}
  function adj_dict_find(dict, key, value) result(ierr)
    type(adj_dictionary), intent(inout) :: dict
    character(len=*), intent(in) :: key
    character(len=*), intent(out) :: value
    integer :: ierr
  end function adj_dict_find
\end{fortrancode}
\end{minipage}
\end{boxwithtitle}

Finds an entry in the dictionary \texttt{dict} with key \texttt{key}.

If no corresponding key exist in the dictionary, the function returns \refapi{ADJ_ERR_DICT_FAILED}.
Otherwise, the value can be accessed with \texttt{value}.



\defapis{adj_dict_print}
\begin{boxwithtitle}{\texttt{adj_dict_print}}
\begin{minipage}{\columnwidth}
\begin{ccode}
  void adj_dict_print(adj_dictionary* dict);
\end{ccode}
\begin{fortrancode}
  subroutine adj_dict_print(dict)
    type(adj_dictionary), intent(inout) :: dict
  end subroutine adj_dict_print
\end{fortrancode}
\end{minipage}
\end{boxwithtitle}

Prints the key $\to$ value list in \texttt{dict} to the standard output. 

\defapis{adj_dict_destroy}
\begin{boxwithtitle}{\texttt{adj_dict_destroy}}
\begin{minipage}{\columnwidth}
\begin{ccode}
  int adj_dict_destroy(adj_dictionary* dict);
\end{ccode}
\begin{fortrancode}
  function adj_dict_destroy(dict) result(ierr)
    type(adj_dictionary), intent(inout) :: dict
    integer(kind=c_int) :: ierr
  end function adj_dict_destroy
\end{fortrancode}
\end{minipage}
\end{boxwithtitle}

Destroys the dictionary \texttt{dict}.

