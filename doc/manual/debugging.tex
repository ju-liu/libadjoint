\chapter{Debugging facilities} \label{chap:debugging}

\begin{synopsis}
\end{synopsis}
\minitoc
\vspace{\fill}
\newpage

\section{Handling return codes}
Almost all public \libadjoint functions return an integer return code,
like PETSc or MPI. After every call to a \libadjoint routine, the developer
should inspect the return code for success or failure. This inspection is
performed with the \refapi{adj_chkierr} routine.
\defapis{adj_chkierr}
\begin{boxwithtitle}{\texttt{adj_chkierr}}
\begin{minipage}{\columnwidth}
\begin{ccode}
  void adj_chkierr(int ierr);
\end{ccode}
\begin{fortrancode}
  subroutine adj_chkierr(ierr)
    integer(kind=c_int), intent(in), value :: ierr
  end subroutine adj_chkierr
\end{fortrancode}
\end{minipage}
\end{boxwithtitle}
This routine inspects the return code from other \libadjoint functions,
and possibly prints out useful debugging information. If the return code
indicates success, this routine does nothing. If the return code indicates
a warning, this routine prints out the error message set by \libadjoint,
and returns. If the return code indicates an error, this routine prints out the
error message set by \libadjoint, and aborts execution.

A full list of the possible return codes is given in \autoref{chap:return_codes}.
\section{HTML output of the adjointer state}
\defapis{adj_adjointer_to_html}
\begin{boxwithtitle}{\texttt{adj_adjointer_to_html}}
\begin{minipage}{\columnwidth}
\begin{ccode}
  int adj_adjointer_to_html(adj_adjointer* adjointer, char* filename, int type);
\end{ccode}
\begin{fortrancode}
  function adj_adjointer_to_html(adjointer, filename, type) result(ierr)
    type(adj_adjointer), intent(in) :: adjointer
    character(len=*), intent(in) :: filename
    integer(kind=c_int), intent(in), value :: type
    integer(kind=c_int) :: ierr
  end function adj_adjointer_to_html
\end{fortrancode}
\end{minipage}
\end{boxwithtitle}
This routine prints a HTML visualisation of the adjointer state to disk. Currently,
this function supports two constants for \texttt{type}: \texttt{ADJ_FORWARD} and
\texttt{ADJ_ADJOINT}.
\section{Comparing against the original forward run} \label{sec:compare}
As documented in section \autoref{sec:replay}, \libadjoint can replay the forward
run. A very powerful debugging strategy is to replay the forward run, \emph{comparing
each recomputed variable to its value computed during the actual forward model run}.
To facilitate this, \libadjoint offers two functions that act on the \refapi{adj_storage_data}
type before it is passed to \refapi{adj_record_variable}.
\defapis{adj_storage_set_compare}
\begin{boxwithtitle}{\texttt{adj_storage_set_compare}}
\begin{minipage}{\columnwidth}
\begin{ccode}
  int adj_storage_set_compare(adj_storage_data* data, int compare, 
                              adj_scalar comparison_tolerance);
\end{ccode}
\begin{fortrancode}
  function adj_storage_set_compare(data, compare, comparison_tolerance)
           result(ierr)
    type(adj_storage_data), intent(inout) :: data
    logical, intent(in) :: compare
    adj_scalar_f, intent(in) :: comparison_tolerance
    integer(kind=c_int) :: ierr
  end function adj_storage_set_compare
\end{fortrancode}
\end{minipage}
\end{boxwithtitle}
If this flag is set to \texttt{ADJ_TRUE} (C) or \texttt{.true.} (Fortran),
then \refapi{adj_record_variable} will compare the value supplied with any value
that had been previously recorded. To ``compare'' means to compute
\begin{equation*}
\texttt{diff} = \left|\left| v_{\textrm{new}} - v_{\textrm{old}} \right|\right|,
\end{equation*}
and return a warning \refapi{ADJ_WARN_COMPARISON_FAILED} if \texttt{diff} is greater
than the specified \texttt{comparison_tolerance}.

This feature is most useful if the forward model has been run in a debugging mode that
recorded the value of every variable computed using \refapi{adj_record_variable}. Then,
with this feature, \libadjoint can compare its interpretation of the forward model against
the truth, and issue warnings on any differences that inform the model developer exactly
where the annotation and the model differ.

The use of this feature requires the \refapi{ADJ_VEC_GET_NORM_CB} data callback.

\defapis{adj_storage_set_overwrite}
\begin{boxwithtitle}{\texttt{adj_storage_set_overwrite}}
\begin{minipage}{\columnwidth}
\begin{ccode}
  int adj_storage_set_overwrite(adj_storage_data* data, int overwrite);
\end{ccode}
\begin{fortrancode}
  function adj_storage_set_overwrite(data, overwrite)
           result(ierr)
    type(adj_storage_data), intent(inout) :: data
    logical, intent(in) :: overwrite
    integer(kind=c_int) :: ierr
  end function adj_storage_set_overwrite
\end{fortrancode}
\end{minipage}
\end{boxwithtitle}
If this flag is set to \texttt{ADJ_TRUE} (C) or \texttt{.true.} (Fortran),
then \refapi{adj_record_variable} will overwrite any value previously recorded
with the value supplied. Without this flag, \refapi{adj_record_variable} will
return an \refapi{ADJ_WARN_ALREADY_RECORDED} and not replace the previous value.

This feature is useful when comparing the forward replay against the original
model run. If the forward replay is very slighly wrong, it may pass a comparison
activated with \refapi{adj_storage_set_compare}, but this initial error will
not propagate unless the new (incorrect) value is used instead of the old (correct)
value in previous calculations.

\section{Checking the Hermitian of operators}
One common class of bugs is implementing the Hermitian of 
each operator that features in the forward annotation. Typically,
the Hermitian case will have to be written specially, as the forward
model will not use it. If the operator is explicitly available as a matrix,
computing its Hermitian action is straightforward; however, if the operator is never
stored, then computing its Hermitian action may not be so easy. Therefore,
it is very useful to have a facility to automatically check the consistency
of the Hermitian and non-Hermitian case.

The idea of the Hermitian check is to compute both sides of the identity
\begin{equation*}
\langle y, Ax \rangle = \langle A^* y, x \rangle,
\end{equation*}
and check that they are equal to within some tolerance. In the
\refapi{adj_block_set_test_hermitian} case, $A$ is the block
itself; in the \refapi{adj_nonlinear_block_set_test_hermitian}
case, $A$ is the Jacobian of the nonlinear block with respect to its nonlinear
dependency.

\defapis{adj_block_set_test_hermitian}
\begin{boxwithtitle}{\texttt{adj_block_set_test_hermitian}}
\begin{minipage}{\columnwidth}
\begin{ccode}
  int adj_block_set_test_hermitian(adj_block* block, int test_hermitian,
                                   int number_of_tests,
                                   adj_scalar tolerance);
\end{ccode}
\begin{fortrancode}
  function adj_block_set_test_hermitian(block, test_hermitian,
                                        number_of_tests, tolerance)
                                        result(ierr)
    type(adj_block), intent(inout) :: block
    logical, intent(in) :: test_hermitian
    integer(kind=c_int), intent(in), value :: number_of_tests
    adj_scalar_f, intent(in), value :: tolerance
    integer(kind=c_int) :: ierr
  end function adj_block_set_test_hermitian
\end{fortrancode}
\end{minipage}
\end{boxwithtitle}
If this flag is set to \texttt{ADJ_TRUE} (C) or \texttt{.true.} (Fortran),
then when the action of this block is evaluated, \libadjoint will also perform
a Hermitian consistency test for a block $A$. It will populate \texttt{number_of_tests}
random vectors $x, y$, and check both sides of the identity
\begin{equation*}
\langle y, Ax \rangle = \langle A^* y, x \rangle,
\end{equation*}
to within a tolerance \texttt{tolerance}.

If the comparison fails, \libadjoint
will return an \refapi{ADJ_ERR_TOLERANCE_EXCEEDED}.

The use of this feature requires the \refapi{ADJ_VEC_SET_RANDOM_CB} and
\refapi{ADJ_VEC_DOT_PRODUCT_CB} data callbacks.

\defapis{adj_nonlinear_block_set_test_hermitian}
\begin{boxwithtitle}{\texttt{adj_nonlinear_block_set_test_hermitian}}
\begin{minipage}{\columnwidth}
\begin{ccode}
  int adj_nonlinear_block_set_test_hermitian(adj_nonlinear_block* nblock, int test_hermitian, 
                                   int number_of_tests, adj_scalar tolerance)
\end{ccode}
\begin{fortrancode}
  function adj_nonlinear_block_set_test_hermitian(nblock, test_hermitian, 
                                     number_of_tests, tolerance) result(ierr)
    type(adj_nonlinear_block), intent(inout) :: nblock
    logical, intent(in) :: test_hermitian
    integer(kind=c_int), intent(in), value :: number_of_tests
    adj_scalar_f, intent(in), value :: tolerance
    integer(kind=c_int) :: ierr
  end function adj_nonlinear_block_set_test_hermitian
\end{fortrancode}
\end{minipage}
\end{boxwithtitle}
If this flag is set to \texttt{ADJ_TRUE} (C) or \texttt{.true.} (Fortran),
then when the action of this nonlinear block is evaluated, \libadjoint will also perform
a Hermitian consistency test for a block $A$. It will populate \texttt{number_of_tests}
random vectors $x, y$, and check both sides of the identity
\begin{equation*}
\langle y, Ax \rangle = \langle A^* y, x \rangle,
\end{equation*}
to within a tolerance \texttt{tolerance}.

If the comparison fails, \libadjoint
will return an \refapi{ADJ_ERR_TOLERANCE_EXCEEDED}.

The use of this feature requires the \refapi{ADJ_VEC_SET_RANDOM_CB} and
\refapi{ADJ_VEC_DOT_PRODUCT_CB} data callbacks.

\section{Checking gradient correctness} \label{sec:derivative_test}
\defapis{adj_nonlinear_block_set_test_derivative}

\section{\texttt{adj_adjointer} consistency check}
\defapis{adj_adjointer_check_consistency}
\begin{boxwithtitle}{\texttt{adj_adjointer_check_consistency}}
\begin{minipage}{\columnwidth}
\begin{ccode}
  int adj_adjointer_check_consistency(adj_adjointer* adjointer);
\end{ccode}
\begin{fortrancode}
  function adj_adjointer_check_consistency(adjointer) result(ierr)
    type(adj_adjointer), intent(in) :: adjointer
    integer(kind=c_int) :: ierr
  end function adj_adjointer_check_consistency
\end{fortrancode}
\end{minipage}
\end{boxwithtitle}
This routine runs miscellaneous debugging checks to ensure the validity
of a given annotation. For example, it checks that all non-auxiliary
variables seen by \libadjoint have an equation set and that the system
presented is lower triangular.
