\chapter{Debugging facilities} \label{chap:debugging}

\begin{synopsis}
\end{synopsis}
\minitoc
\vspace{\fill}
\newpage

\section{Handling return codes}
Almost all public \libadjoint functions return an integer return code,
like PETSc or MPI. After every call to a \libadjoint routine, the developer
should inspect the return code for success or failure. This inspection is
performed with the \refapi{adj_chkierr} routine.
\defapis{adj_chkierr}
\begin{boxwithtitle}{\texttt{adj_chkierr}}
\begin{minipage}{\columnwidth}
\begin{ccode}
  void adj_chkierr(int ierr);
\end{ccode}
\begin{fortrancode}
  subroutine adj_chkierr(ierr)
    integer(kind=c_int), intent(in), value :: ierr
  end subroutine adj_chkierr
\end{fortrancode}
\end{minipage}
\end{boxwithtitle}
This routine inspects the return code from other \libadjoint functions,
and possibly prints out useful debugging information. If the return code
indicates success, this routine does nothing. If the return code indicates
a warning, this routine prints out the error message set by \libadjoint,
and returns. If the return code indicates an error, this routine prints out the
error message set by \libadjoint, and aborts execution.

A full list of the possible return codes is given in \autoref{chap:return_codes}.
\section{HTML output of the adjointer state}
\defapis{adj_adjointer_to_html}
\begin{boxwithtitle}{\texttt{adj_adjointer_to_html}}
\begin{minipage}{\columnwidth}
\begin{ccode}
  int adj_adjointer_to_html(adj_adjointer* adjointer, char* filename, int type);
\end{ccode}
\begin{fortrancode}
  function adj_adjointer_to_html(adjointer, filename, type) result(ierr)
    type(adj_adjointer), intent(in) :: adjointer
    character(len=*), intent(in) :: filename
    integer(kind=c_int), intent(in), value :: type
    integer(kind=c_int) :: ierr
  end function adj_adjointer_to_html
\end{fortrancode}
\end{minipage}
\end{boxwithtitle}
This routine prints a HTML visualisation of the adjointer state to disk. Currently,
this function supports two constants for \texttt{type}: \texttt{ADJ_FORWARD} and
\texttt{ADJ_ADJOINT}.
\section{Comparing against the original forward run} \label{sec:replay}
\defapis{adj_storage_set_compare}
\defapis{adj_storage_set_overwrite}
\section{Checking operator transposition}
\defapis{adj_block_set_test_hermitian}
\defapis{adj_nonlinear_block_set_test_hermitian}
\section{Checking gradient correctness} \label{sec:derivative_test}
\defapis{adj_nonlinear_block_set_test_derivative}
\section{\texttt{adj_adjointer} consistency check}
\defapis{adj_adjointer_check_consistency}
