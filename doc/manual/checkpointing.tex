\chapter{Checkpointing}\label{chap:checkpointing}

\begin{synopsis}
\end{synopsis}
\minitoc
\vspace{\fill}
\newpage

\section{Introduction}
\libadjoint is distributed with Revolve, a library for optimal checkpointing for adjoint calculations. 
\section{Initialisation}


\defapis{adj_set_checkpoint_strategy}
\begin{boxwithtitle}{Function interface for \texttt{adj_set_checkpoint_strategy}}
\begin{minipage}{\columnwidth}
\begin{ccode}
  int adj_set_checkpoint_strategy(adj_adjointer* adjointer, int strategy);
\end{ccode}
\begin{fortrancode}
function adj_set_checkpoint_strategy(adjointer, strategy) result(ierr) 
  type(adj_adjointer), intent(inout) :: adjointer
  integer(kind=c_int), intent(in), value :: strategy
  integer(kind=c_int) :: ierr
end function adj_set_checkpoint_strategy
\end{fortrancode}
\end{minipage}
\end{boxwithtitle}

This function defines the checkpointing strategy to be used. 
The options for the argument \texttt{strategy} are:
\begin{itemize}
\item \texttt{ADJ_CHECKPOINT_NONE}: No checkpointing. The default.
\item \texttt{ADJ_CHECKPOINT_REVOLVE_OFFLINE}: Uses Revolve in offline checkpointing mode.
\item \texttt{ADJ_CHECKPOINT_REVOLVE_MULTISTAGE}: Uses Revolve in multistage checkpointing mode.
\item \texttt{ADJ_CHECKPOINT_REVOLVE_ONLINE}: Uses Revolve in online checkpointing mode.
\end{itemize}


\defapis{adj_set_revolve_options}
\begin{boxwithtitle}{Function interface for \texttt{adj_set_revolve_options}}
\begin{minipage}{\columnwidth}
\begin{ccode}
int adj_set_revolve_options(adj_adjointer* adjointer, int steps, int snaps_on_disk, 
                            int snaps_in_ram, int verbose);
\end{ccode}
\begin{fortrancode}
function adj_set_revolve_options(adjointer, steps, snaps_on_disk, snaps_in_ram, verbose) 
         result(ierr) 
  type(adj_adjointer), intent(inout) :: adjointer
  integer(kind=c_int), intent(in) :: steps 
  integer(kind=c_int), intent(in) :: snaps_on_disk
  integer(kind=c_int), intent(in) :: snaps_in_ram 
  logical, intent(in) :: verbose 
  integer(kind=c_int) :: ierr
end function adj_set_revolve_options
\end{fortrancode}
\end{minipage}
\end{boxwithtitle}

This function sets the checkpointing options for revolve based checkpointing schemes: \texttt{ADJ_CHECKPOINT_REVOLVE_OFFLINE}, \texttt{ADJ_CHECKPOINT_REVOLVE_MULTISTAGE} and \texttt{ADJ_CHECKPOINT_REVOLVE_ONLINE}. 
\texttt{steps} is the total number of timesteps of the simulation (if known).
\texttt{snaps_on_disk} specifies the number of disk checkpoint slots that \libadjoint may use.
\texttt{snaps_in_ram} denotes the number checkpoints that may be stored in memory. 
Hence the total number of checkpoint stored used is $\texttt{snaps_on_disk}+\texttt{snaps_in_ram}$

Depending on the chosen checkpointing strategy, the function arguments must fulfill these properties:
\begin{itemize}
\item \texttt{ADJ_CHECKPOINT_REVOLVE_OFFLINE}: $\texttt{steps}>1$, $\texttt{snaps_in_ram}>0$ and at least two checkpoints.
\item \texttt{ADJ_CHECKPOINT_REVOLVE_MULTISTAGE}: $\texttt{steps}>1$, $\texttt{snaps_in_ram}>0$ and $\texttt{snaps\_on\_disk} \ge 0$.
\item \texttt{ADJ_CHECKPOINT_REVOLVE_ONLINE}: $\texttt{snaps_in_ram}>0$ and at least two checkpoints.
\end{itemize}

If the flag \texttt{verbose} is set, \libadjoint will print revolve specific output to the screen. 

\defapis{adj_set_revolve_debug_options}
\begin{boxwithtitle}{Function interface for \texttt{adj_set_revolve_debug_options}}
\begin{minipage}{\columnwidth}
\begin{ccode}
int adj_set_revolve_debug_options(adj_adjointer* adjointer, int overwrite, 
                                  adj_scalar comparison_tolerance);
\end{ccode}
\begin{fortrancode}
function adj_set_revolve_debug_options(adjointer, overwrite, comparison_tolerance) 
                                      result(ierr)
  type(adj_adjointer), intent(inout) :: adjointer
  logical, intent(in) :: overwrite
  adj_scalar_f, intent(in) :: comparison_tolerance
end function adj_set_revolve_debug_options
\end{fortrancode}
\end{minipage}
\end{boxwithtitle}

This function sets the debugging options for checkpointing. Its parameters are: 
\begin{itemize}
\item \texttt{overwrite}: This flag influences the behaviour of the internal replays of the checkpointing scheme. If \texttt{ADJ_TRUE}, then forward equations during a replay are solved even if a solution is already recorded. The computed value is then compared with the existing one, and prints a warning in case a specified tolerance is exceeded.
\item \texttt{comparison_tolerance}: This parameter is only used when \texttt{overwrite=ADJ_TRUE}. It specifies the tolerance for comparing recomputed with previously computed values.
\end{itemize}
The \texttt{overwrite} feature is used to ensures that the forward solutions and the checkpoint replay solutions are consistent.

By default, i.e. if \refapi{adj_set_revolve_debug_options} is never called, \texttt{overwrite} is set to \texttt{ADJ_FALSE}.

\defapis{adj_check_checkpoints}
\begin{boxwithtitle}{Function interface for \texttt{adj_adjointer_check_checkpoints}}
\begin{minipage}{\columnwidth}
\begin{ccode}
int adj_adjointer_check_checkpoints(adj_adjointer* adjointer);
\end{ccode}
\begin{fortrancode}
function adj_adjointer_check_checkpoints(adjointer) result(ierr) 
  type(adj_adjointer), intent(in) :: adjointer
  integer(kind=c_int) :: ierr
end function adj_adjointer_check_checkpoints
\end{fortrancode}
\end{minipage}
\end{boxwithtitle}

This function checks that the validity of all checkpoints, i.e. that the required variables for restarting the simulation at the checkpoint equation are correctly recorded.
It should be called just before starting the adjoint main loop.
If the return value is \texttt{ADJ_OK}, then the required information is available for revolve to solve the adjoint system.


\defapis{adj_storage_set_checkpoint}
\begin{boxwithtitle}{Function interface for \texttt{adj_storage_set_checkpoint}}
\begin{minipage}{\columnwidth}
\begin{ccode}
int adj_storage_set_checkpoint(adj_storage_data* data, int checkpoint)
\end{ccode}
\begin{fortrancode}
function adj_storage_set_checkpoint(data, checkpoint) result(ierr)
  type(adj_storage_data), intent(inout) :: data
  logical, intent(in) :: checkpoint
  integer(kind=c_int) :: ierr
end function adj_storage_set_checkpoint
\end{fortrancode}
\end{minipage}
\end{boxwithtitle}

This function marks a variable to be a checkpoint variable. This should be used for all variables that are required to restart from a checkpoint.


