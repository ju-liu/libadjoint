\chapter{Checkpointing}

\begin{synopsis}
\end{synopsis}
\minitoc
\vspace{\fill}
\newpage

\section{Introduction}
\libadjoint is distributed with Revolve, a library for optimal checkpointing for adjoint calculations. 
\section{Initialisation}


\defapis{adj_set_checkpoint_strategy}
\begin{boxwithtitle}{Function interface for \texttt{adj_set_checkpoint_strategy}}
\begin{minipage}{\columnwidth}
\begin{ccode}
  int adj_set_checkpoint_strategy(adj_adjointer* adjointer, int strategy);
\end{ccode}
\begin{fortrancode}
function adj_set_checkpoint_strategy(adjointer, strategy) result(ierr) 
  type(adj_adjointer), intent(inout) :: adjointer
  integer(kind=c_int), intent(in), value :: strategy
  integer(kind=c_int) :: ierr
end function adj_set_checkpoint_strategy
\end{fortrancode}
\end{minipage}
\end{boxwithtitle}

This function sets the checkpointing strategy to be used. 
The options for the argument \texttt{strategy} are:
\begin{itemize}
\item \texttt{ADJ_CHECKPOINT_NONE}: No checkpointing. The default.
\item \texttt{ADJ_CHECKPOINT_REVOLVE_OFFLINE}: Uses Revolve in offline checkpointing mode.
\item \texttt{ADJ_CHECKPOINT_REVOLVE_MULTISTAGE}: Uses Revolve in multistage checkpointing mode.
\item \texttt{ADJ_CHECKPOINT_REVOLVE_ONLINE}: Uses Revolve in online checkpointing mode.
\end{itemize}


\defapis{adj_set_revolve_options}
\begin{boxwithtitle}{Function interface for \texttt{adj_set_revolve_options}}
\begin{minipage}{\columnwidth}
\begin{ccode}
int adj_set_revolve_options(adj_adjointer* adjointer, int steps, int snaps, 
                            int snaps_in_ram, int verbose);
\end{ccode}
\begin{fortrancode}
function adj_set_revolve_options(adjointer, steps, snaps, snaps_in_ram, verbose) 
         result(ierr) 
  type(adj_adjointer), intent(inout) :: adjointer
  integer(kind=c_int), intent(in) :: steps 
  integer(kind=c_int), intent(in) :: snaps 
  integer(kind=c_int), intent(in) :: snaps_in_ram 
  logical, intent(in) :: verbose 
  integer(kind=c_int) :: ierr
end function adj_set_revolve_options
\end{fortrancode}
\end{minipage}
\end{boxwithtitle}

This function sets the checkpointing options if using a revolve checkpointing scheme. 
\texttt{steps} is the total number of timesteps of the simulation.
\texttt{snaps} specifies the overall number of checkpoint slots that \libadjoint may use.
\texttt{snaps_in_ram} denotes the number checkpoints that may be stored in memory. 
Hence the number of checkpoint stored on disk is $\texttt{snaps}-\texttt{snaps_in_ram}$

Depending on the chosen checkpointing strategy, these options must fulfill these properties:
\begin{itemize}
\item \texttt{ADJ_CHECKPOINT_REVOLVE_OFFLINE}: $\texttt{steps}>0$ and $\texttt{snaps}>0$.
\item \texttt{ADJ_CHECKPOINT_REVOLVE_MULTISTAGE}: $\texttt{steps}>0$, $\texttt{snaps}>0$, $\texttt{snaps\_in\_ram}\ge0$ and $\texttt{snaps\_in\_ram}\le \texttt{snaps}$.
\item \texttt{ADJ_CHECKPOINT_REVOLVE_ONLINE}: $\texttt{snaps}>0$.
\end{itemize}

If the flag \texttt{verbose} is set, \libadjoint will print revolve specific output to the screen. 


\defapis{adj_check_checkpoints}

If this check passes, then all information is available for revolve to solve the adjoint system.
