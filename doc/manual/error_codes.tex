\chapter{Return codes}

All public functions offered by \libadjoint return a return code indicating
success or failure. The only exception to this is \refapi{adj_chkierr}, which
takes in that return code and prints an informative message if the function was
not successful.

\defapi{ADJ_OK}
\texttt{ADJ_OK} is the expected return code, which indicates that the function
worked as expected.

\section{Warnings}
Warnings are indicated by a negative return code. When passed to \refapi{adj_chkierr},
a warning message will be printed, but execution will continue as normal.

\defapis{ADJ_WARN_ALREADY_RECORDED}
If you attempt to record a variable with \refapi{adj_record_variable}, and that variable
already has a value recorded, then \refapi{ADJ_WARN_ALREADY_RECORDED} will be returned.
Note that if you wish to overwrite a previously-recorded value, set the overwrite flag
with \refapi{adj_storage_set_overwrite}.

\defapis{ADJ_WARN_COMPARISON_FAILED}
When recording a variable which has already has a recorded value, it is possible to activate the
compare flag with \refapi{adj_storage_set_compare}. When this is detected by \refapi{adj_record_variable},
it will compare the recorded value against the newly supplied one; if they differ by more than
the developer-specified tolerance, this warning is issued. The primary purpose of this is to detect
inconsistencies between the original forward run and the forward replay run through \libadjoint,
as discussed in \autoref{sec:replay}.

This warning can also be issued by \refapi{adj_get_forward_equation} or \refapi{adj_get_adjoint_equation},
if the derivative test flag has been set on a nonlinear block with \refapi{adj_nonlinear_block_set_test_derivative}.
As discussed in \autoref{sec:derivative_test}, this flag indicates that \libadjoint should apply the derivative
test to the supplied derivative routine. In the derivative test, the finite difference error should converge
at first order, whereas when corrected with the gradient it should converge at second order. The derivative test
compares both of these numbers to their expected values, and if they differ, issues a \refapi{ADJ_WARN_COMPARISON_FAILED}.\index{algorithmic differentiation!derivative test}
In particular, if your supplied derivative code is incorrect, then this warning will be issued.

\defapis{ADJ_WARN_UNINITIALISED_VALUE}
This warning is issued by \refapi{adj_evaluate_functional} if no dependencies have been
registered for that particular functional at that timestep. A functional by definition must depend on the solution. To fix this, you must register the dependencies with
\refapi{adj_timestep_set_functional_dependencies}.

\defapis{ADJ_WARN_NOT_IMPLEMENTED}
This warning is issued when a na\"ive algorithm is used in place of a more complex,
faster algorithm that is not yet implemented. At present, the only routine where
this is used is in an internal function called \texttt{adj_simplify_derivatives}, which uses a simple quadratic
algorithm in place of a linear algorithm that is much more difficult to code.
However, the quadratic algorithm should be sufficiently fast for all realistic
use cases, and so developers should never see this.

\section{Errors}
Errors are indicated by a positive return code. When passed to \refapi{adj_chkierr},
an error message will be printed, and execution will stop.

\defapis{ADJ_ERR_INVALID_INPUTS}
All \libadjoint routines do as much error checking as possible
on the developer-supplied inputs. This error is issued whenever a \libadjoint routine
detects that some input is invalid. For example, if the developer attempts to create
an \refapi{adj_variable} with a negative iteration number, this error will be issued. Or
if a hermitian argument is neither \texttt{ADJ_TRUE} nor \texttt{ADJ_FALSE}, then this warning
will be issued. For more details, read the error message printed by \refapi{adj_chkierr}.

\defapis{ADJ_ERR_HASH_FAILED}
At the core of an \refapi{adj_adjointer}, there is a hash table that maps an
\refapi{adj_variable} to the data that the \refapi{adj_adjointer} records about it.
This table is referenced for almost every operation that \libadjoint performs. This error
is issued if an \refapi{adj_variable} has been supplied that the hash table was
unable to find. Check all \refapi{adj_variable} inputs.

\defapis{ADJ_ERR_NEED_CALLBACK}
This error is issued when \libadjoint wants to perform an operation, but does
not have the necessary function to do it. See \autoref{chap:callbacks}. You
must register a function pointer for the necessary callback with \refapi{adj_register_data_callback},
\refapi{adj_register_operator_callback} or similar.

\defapis{ADJ_ERR_NEED_VALUE}
This error is issued when \libadjoint needs a value for a variable, but
does not have one available. Common examples are: it is necessary to assemble
a nonlinear operator but the dependencies are not available; it is necessary to compute
the derivative of the functional but the dependencies are not available; in order to
assemble an adjoint equation, a block must be applied to a previously computed
adjoint solution, but it has not been recorded. Record the value of the variable
with \refapi{adj_record_variable}.

\defapis{ADJ_ERR_NOT_IMPLEMENTED}
This error is issued when a feature which is necessary for the correct assembly
of the adjoint equations has not yet been implemented. See \autoref{chap:todo}
for the current list of unimplemented features. Development assistance on
\libadjoint is always welcome!

\defapis{ADJ_ERR_DICT_FAILED}
This error is issued when \refapi{adj_dict_find} is called on an \refapi{adj_dict}
which does not have the relevant key.

\defapis{ADJ_ERR_TOLERANCE_EXCEEDED}
This error is issued if the transpose test fails. The transpose test is activated
with the functions \\\refapi{adj_block_set_test_hermitian} and \refapi{adj_nonlinear_block_set_test_hermitian}.
This failure\\ means that the developer has an inconsistency between the \texttt{hermitian == ADJ_TRUE}
and \\\texttt{hermitian == ADJ_FALSE} cases in an operator callback.

\defapis{ADJ_ERR_MALLOC_FAILED}
\libadjoint dynamically allocates memory to store information that the developer
has passed. Every time \libadjoint allocates memory with \texttt{malloc} and
similar functions, it checks that the memory allocation was successful. If the memory
allocation was not successful, this error is issued. Note that the raising of this
issue may leave the \refapi{adj_adjointer} in an undefined state; further work would
be required to ensure that the \refapi{adj_adjointer} is always in a consistent state
regardless of memory allocation failure, but this error is so infrequent that 
this improvement is not a \libadjoint development priority.
