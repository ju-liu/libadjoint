\chapter{Return codes}

All public functions offered by \libadjoint return a return code indicating
success or failure. The only exception to this is \refapi{adj_chkierr}, which
takes in that return code and prints an informative message if the function was
not successful.

\defapi{ADJ_OK}
\texttt{ADJ_OK} is the expected return code, which indicates that the function
worked as expected.

\section{Warnings}
Warnings are indicated by a negative return code. When passed to \refapi{adj_chkierr},
a warning message will be printed, but execution will continue as normal.

\defapis{ADJ_WARN_ALREADY_RECORDED}
If you attempt to record a variable with \refapi{adj_record_variable}, and that variable
already has a value recorded, then \refapi{ADJ_WARN_ALREADY_RECORDED} will be returned.
Note that if you wish to overwrite a previously-recorded value, set the overwrite flag
with \refapi{adj_storage_set_overwrite}.

\defapis{ADJ_WARN_COMPARISON_FAILED}
When recording a variable which has already has a recorded value, it is possible to activate the
compare flag with \refapi{adj_storage_set_compare}. When this is detected by \refapi{adj_record_variable},
it will compare the recorded value against the newly supplied one; if they differ by more than
the developer-specified tolerance, this warning is issued. The primary purpose of this is to detect
inconsistencies between the original forward run and the forward replay run through \libadjoint,
as discussed in \autoref{sec:replay}.

This warning can also be issued by \refapi{adj_get_forward_equation} or \refapi{adj_get_adjoint_equation},
if the derivative test flag has been set on a nonlinear block with \refapi{adj_nonlinear_block_set_test_derivative}.
As discussed in \autoref{sec:derivative_test}, this flag indicates that \libadjoint should apply the derivative
test to the supplied derivative routine. In the derivative test, the finite difference error should converge
at first order, whereas when corrected with the gradient it should converge at second order. The derivative test
compares both of these numbers to their expected values, and if they differ, issues a \refapi{ADJ_WARN_COMPARISON_FAILED}.\index{algorithmic differentiation!derivative test}
In particular, if your supplied derivative code is incorrect, then this warning will be issued.

\defapis{ADJ_WARN_UNINITIALISED_VALUE}
This warning is issued by \refapi{adj_evaluate_functional} if no dependencies have been
registered for that particular functional at that timestep. A functional by definition must depend on the solution. To fix this, you must register the dependencies with
\refapi{adj_timestep_set_functional_dependencies}.

\defapis{ADJ_WARN_NOT_IMPLEMENTED}
This warning is issued when a na\"ive algorithm is used in place of a more complex,
faster algorithm that is not yet implemented. At present, the only routine where
this is used is in an internal function called \texttt{adj_simplify_derivatives}, which uses a simple quadratic
algorithm in place of a linear algorithm that is much more difficult to code.
However, the quadratic algorithm should be sufficiently fast for all realistic
use cases, and so developers should never see this.

\section{Errors}
Errors are indicated by a positive return code. When passed to \refapi{adj_chkierr},
an error message will be printed, and execution will stop.

\defapis{ADJ_ERR_INVALID_INPUTS}
\defapis{ADJ_ERR_HASH_FAILED}
\defapis{ADJ_ERR_NEED_CALLBACK}
\defapis{ADJ_ERR_NEED_VALUE}
\defapis{ADJ_ERR_NOT_IMPLEMENTED}
\defapis{ADJ_ERR_DICT_FAILED}
\defapis{ADJ_ERR_TOLERANCE_EXCEEDED}
\defapis{ADJ_ERR_MALLOC_FAILED}
