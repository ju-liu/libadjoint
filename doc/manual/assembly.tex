\chapter{Assembling equations}
\begin{synopsis}
\end{synopsis}
\minitoc
\vspace{\fill}
\newpage


This chapter introduces the assembly routines for the adjoint equation 
and gives a typical example for an adjoint main loop.
Furthermore, the routines for assembling the forward equations are discussed which are needed 
for the comparison against the original forward run, see \autoref{sec:compare}.

\section{Assembling the adjoint equations} \label{sec:adjoint_assembly}

\defapis{adj_get_adjoint_equation} 

\begin{boxwithtitle}{\texttt{adj_get_adjoint_equation}}
\begin{minipage}{\columnwidth}
\begin{ccode}
  int adj_get_adjoint_equation(adj_adjointer* adjointer, int equation, 
                               char* functional, adj_matrix* lhs, 
                               adj_vector* rhs, adj_variable* adj_var)
\end{ccode}
\begin{fortrancode}   
  function adj_get_adjoint_equation(adjointer, equation, functional, lhs, rhs, 
                                    adj_var) result(ierr)
    type(adj_adjointer), intent(inout) :: adjointer
    integer(kind=c_int), intent(in), value :: equation
    character(len=*), intent(in) :: functional
    type(adj_matrix), intent(out) :: lhs
    type(adj_vector), intent(out) :: rhs
    type(adj_variable), intent(out) :: adj_var
    integer(kind=c_int) :: ierr
  end function adj_get_adjoint_equation
\end{fortrancode}
\end{minipage}
\end{boxwithtitle}

This is the core function of \libadjoint.
It assembles the adjoint equation with index \texttt{equation} for the functional named \texttt{functional}.
If \texttt{equation} is less than zero or larger or equal to the number of registered equations, \refapi{ADJ_ERR_INVALID_INPUTS} is returned.
The result is a linear system with operator \texttt{lhs} and right hand side \texttt{rhs}. 
Furthermore, the variable that this adjoint equation targets is provided as \texttt{adj_var}.
The solution of the linear system must be recorded with \refapi{adj_record_variable}.

\defapis{adj_forget_adjoint_equation}

\begin{boxwithtitle}{\texttt{adj_forget_adjoint_equation}}
\begin{minipage}{\columnwidth}
\begin{ccode}
  int adj_forget_adjoint_equation(adj_adjointer* adjointer, int equation)
\end{ccode}
\begin{fortrancode}   
  function adj_forget_adjoint_equation(adjointer, equation) result(ierr) 
    type(adj_adjointer), intent(inout) :: adjointer
    integer(kind=c_int), intent(in), value :: equation
    integer(kind=c_int) :: ierr
  end function adj_forget_adjoint_equation
\end{fortrancode}
\end{minipage}
\end{boxwithtitle}

This function is used to remove recorded variables that are not longer be necessary to assemble the remaining adjoint equations. 
More specifically, it forgets all recorded variables that are not needed to assemble the adjoint equations with index lower than \texttt{equation}.


\section{A typical adjoint main loop}\label{sec:typical_adjoint_main_loop}

Once the developer has annotated the model and provided all necessary callbacks to \libadjoint, 
everything is in place to assemble and solve the adjoint equations. 

Libadjoint is designed such that the program loop for solving the adjoint equations is simple
and its structure does not depend on the underlying forward model.
Hence, the example adjoint main loop given at the end of this section is a good starting point for any adjoint main loop implementation with \libadjoint.

A sketch of a straightforward (but very inefficient as we will see soon) adjoint main loop could be:

\begin{minipage}{\columnwidth}
\begin{fortrancode}   
  do i = 0, no_functionals ! Loop over the registered functionals
    do j = no_equations, 0, -1 ! Loop backwards(!) over the registered equations
      call adj_get_adjoint_equation ! Assemble the j'th adjoint equation 
                                    ! for functional i
      call solve ! Solve the adjoint equation 
      call adj_record_variable ! Record the adjoint solution
      !! ... ! Do something useful with the adjoint solution
    end do
  end do
  call adj_forget_adjoint_equation ! Forget all recorded variables
\end{fortrancode}
\end{minipage}

For every registered functional, the adjoint equations are solved backwards in time. 
The core function is \refapi{adj_get_adjoint_equation}, which assembles an adjoint equation for a given functional. 
The solution of the resulting linear system is then recorded with \refapi{adj_record_variable} since it is required for the assembly of the remaining adjoint equations.
The adjoint main loop is finalised by calling \refapi{adj_forget_adjoint_equation} to forget the recorded forward variables. 

Although this adjoint main loop would work, it has one big disadvantage: 
The adjoint loop is iterated independently for every functional. 
Hence the recorded forward variables are requested separately for each functional. 
Depending on the storage strategy, each request might be very expensive, e.g. involve running parts of the forward run again.

A more efficient adjoint main loop implementation is obtained by looping over the adjoint equations first and then over the functionals:

\begin{minipage}{\columnwidth}
\begin{fortrancode}   
  do i = no_equations, 0, -1 ! Loop backwards(!) over the registered equations
    do j = 0, no_functionals ! Loop over the registered functionals
      call adj_get_adjoint_equation ! Assemble the i'th adjoint equation 
                                    ! for functional j
      call solve ! Solve the adjoint equation 
      call adj_record_variable ! Record the adjoint solution
      !! ... ! Do something useful with the adjoint solution
    end do
    call adj_forget_adjoint_equation ! Forget any recorded forward/adjoint 
                                     ! variables that not needed for the 
                                     ! remaining adjoint equations
  end do
\end{fortrancode}
\end{minipage}

Before a full implementation of a typical adjoint main loop can be given, a few helper functions need to be introduced.


\defapis{adj_equation_count}

\begin{boxwithtitle}{\texttt{adj_equation_count}}
\begin{minipage}{\columnwidth}
\begin{ccode}
  int adj_equation_count(adj_adjointer* adjointer, int* count)
\end{ccode}
\begin{fortrancode}   
  function adj_equation_count(adjointer, count) result(ierr) 
    type(adj_adjointer), intent(in) :: adjointer
    integer(kind=c_int), intent(inout) :: count
    integer(kind=c_int) :: ierr
  end function adj_equation_count
\end{fortrancode}
\end{minipage}
\end{boxwithtitle}

This function queries the total number of registered equations in \texttt{adjointer}.
The result is saved in \texttt{count}.



\defapis{adj_timestep_count}

\begin{boxwithtitle}{\texttt{adj_timestep_count}}
\begin{minipage}{\columnwidth}
\begin{ccode}
  int adj_timestep_count(adj_adjointer* adjointer, int* count)
\end{ccode}
\begin{fortrancode}   
  function adj_timestep_count(adjointer, count) result(ierr) 
    type(adj_adjointer), intent(in) :: adjointer
    integer(kind=c_int), intent(out) :: count
    integer(kind=c_int) :: ierr
  end function adj_timestep_count
\end{fortrancode}
\end{minipage}
\end{boxwithtitle}

This function queries the total number of timesteps in \texttt{adjointer}.
The result is saved in \texttt{count}.


\defapis{adj_iteration_count}

\begin{boxwithtitle}{\texttt{adj_iteration_count}}
\begin{minipage}{\columnwidth}
\begin{ccode}
  int adj_iteration_count(adj_adjointer* adjointer, adj_variable variable, 
                          int* count)
\end{ccode}
\begin{fortrancode}   
  function adj_iteration_count(adjointer, variable, count) result(ierr) 
    type(adj_adjointer), intent(in) :: adjointer
    type(adj_variable), intent(in), value :: variable
    integer(kind=c_int), intent(out) :: count
    integer(kind=c_int) :: ierr
  end function adj_iteration_count
\end{fortrancode}
\end{minipage}
\end{boxwithtitle}

This function queries the number iterations for the variable in \refapi{adj_adjointer} that matches the name, timestep and the auxiliary flag of the model variable \texttt{variable}.
If no iteration is found an \refapi{ADJ_ERR_INVALID_INPUTS} error code is returned.
The result is saved in \texttt{count}.


\defapis{adj_timestep_start_equation}

\begin{boxwithtitle}{\texttt{adj_timestep_start_equation}}
\begin{minipage}{\columnwidth}
\begin{ccode}
  int adj_timestep_start_equation(adj_adjointer* adjointer, int timestep, 
                                  int* start)
\end{ccode}
\begin{fortrancode}   
  function adj_timestep_start_equation(adjointer, timestep, start) result(ierr) 
    type(adj_adjointer), intent(in) :: adjointer
    integer(kind=c_int), intent(in), value :: timestep
    integer(kind=c_int), intent(out) :: start
    integer(kind=c_int) :: ierr
  end function adj_timestep_start_equation
\end{fortrancode}
\end{minipage}
\end{boxwithtitle}

This function returns the index of the first registered equation for the supplied timstep.
It is commonly used in combination with \refapi{adj_timestep_end_equation} and \refapi{adj_get_adjoint_equation} (or \refapi{adj_get_forward_equation}) to loop over every equation of one timestep.



\defapis{adj_timestep_end_equation}

\begin{boxwithtitle}{\texttt{adj_timestep_end_equation}}
\begin{minipage}{\columnwidth}
\begin{ccode}
  int adj_timestep_end_equation(adj_adjointer* adjointer, int timestep, int* end)
\end{ccode}
\begin{fortrancode}   
  function adj_timestep_end_equation(adjointer, timestep, end) result(ierr) 
    type(adj_adjointer), intent(in) :: adjointer
    integer(kind=c_int), intent(in), value :: timestep
    integer(kind=c_int), intent(out) :: end
    integer(kind=c_int) :: ierr
  end function adj_timestep_end_equation
\end{fortrancode}
\end{minipage}
\end{boxwithtitle}

This function returns the index of the last registered equation for the supplied timstep.
It is commonly used in combination with \refapi{adj_timestep_start_equation} and \refapi{adj_get_adjoint_equation} (or \refapi{adj_get_forward_equation}) to loop over every equation in a timestep.





\defapis{adj_timestep_get_times}

\begin{boxwithtitle}{\texttt{adj_timestep_get_times}}
\begin{minipage}{\columnwidth}
\begin{ccode}
  int adj_timestep_get_times(adj_adjointer* adjointer, int timestep, 
                             adj_scalar* start, adj_scalar* end)
\end{ccode}
\begin{fortrancode}   
  function adj_timestep_get_times(adjointer, timestep, start, end) result(ierr) 
    type(adj_adjointer), intent(in) :: adjointer
    integer(kind=c_int), intent(in), value :: timestep
    adj_scalar_f, intent(out) :: start
    adj_scalar_f, intent(out) :: end
    integer(kind=c_int) :: ierr
  end function adj_timestep_get_times
\end{fortrancode}
\end{minipage}
\end{boxwithtitle}

This function queries the start and end simulation time of the supplied timestep. 
For this function to work, the start and end time must have been previously defined with \refapi{adj_timestep_set_times}.
If that is not the case an \refapi{ADJ_ERR_INVALID_INPUTS} error code is returned.




\subsection{The full implementation template}
The following code implements a typical adjoint main loop and is intended to be a starting point 
for new adjoint code implementations.

The structure of the loop is based on what was introduced in the beginning of \autoref{sec:typical_adjoint_main_loop}, with the difference that it loops over the adjoint equations of each timestep separately.
This allows to perform certain tasks to be executed at the beginning or end of every adjoint timestep. 

\begin{boxwithtitle}{\texttt{A typical adjoint main loop (1)}}
\begin{minipage}{\columnwidth}
\begin{fortrancode}   
  ! This example assumes that the forward model is correctly annotated,
  ! and all necessary callbacks have been registered. 

  integer :: ierr ! Contains the error code after every libdajoint library call.
  type(adj_adjointer)  :: adjointer ! The adjointer which was used to annotate 
                                    ! the forward model.
  integer :: no_timesteps, timestep ! The total number of timesteps and an 
                                    ! iterator for the timestep loop.
  integer :: start_timestep, end_timestep ! The index of the first and last 
                                          ! equation within in the current 
                                          ! timestep. 
  integer :: no_functionals, functional ! The total number of registered 
                                        ! functionals and an iterator for the 
                                        ! functional loop.
  character(len=ADJ_NAME_LEN), dimension(no_functionals) :: functional_name 
        ! An array of strings containing the name of the registered functional.

  type(adj_vector) :: rhs ! The right hand side of the adjoint equation.
  type(adj_vector) :: soln ! The solution vector of the adjoint equation.
  type(adj_matrix) :: lhs ! The left hand side matrix of the adjoint equation.
  type(adj_variable) :: adj_var ! The target variable of the adjoint equation.
  type(adj_storage_data) :: storage ! The storage object used to record the 
                                    ! adjoint solution.

  ! Get the number of registered timesteps .
  ierr = adj_timestep_count(adjointer, no_timesteps)
  call adj_chkierr(ierr)

  ! The adjoint timeloop runs backward in time.
  do timestep=no_timesteps-1,0,-1

    ! Get the index of the first and last equation of this timestep.
    ierr = adj_timestep_start_equation(adjointer, timestep, start_timestep)
    call adj_chkierr(ierr)

    ierr = adj_timestep_end_equation(adjointer, timestep, end_timestep)
    call adj_chkierr(ierr)

    ! Begin of functional loop
    ! For every functional, the adjoint equations have to be solved seperately.
    do functional=0,no_functionals-1

      ! Begin of equation loop
      ! As in the timestep loop, the equations are solved backwards.
      do equation=end_timestep,start_timestep,-1
      
\end{fortrancode}
\end{minipage}
\end{boxwithtitle}

\begin{boxwithtitle}{\texttt{A typical adjoint main loop (2)}}
\begin{minipage}{\columnwidth}
\begin{fortrancode}   
        ! Let libadjoint assemble the adjoint equation for the current 
        ! functional and equation. The result is a linear system with 
        ! left hand side lhs and right hand side rhs. The variable 
        ! this equation targets is adj_var. 
        ierr = adj_get_adjoint_equation(adjointer, equation, 
                                        trim(functional_names(functional)), 
                                        lhs, rhs, adj_var)
        call adj_chkierr(ierr)

        ! Depending on lhs%klass and rhs%klass one might use different 
        ! solver strategies to solve the adjoint equation.

        ! Solve lhs . adjoint = rhs 
        call solve(soln, lhs, rhs)

        ! Record the adjoint solution. 
        ! In this example, a copy of the solution vector is stored in memory.
        ierr = adj_storage_memory_copy(soln, storage)
        call adj_chkierr(ierr)

        ierr = adj_record_variable(adjointer, adj_var, storage)
        call adj_chkierr(ierr)

        ! Use the adjoint variable
        ! ...

        ! Destroy lhs and rhs using the appropriate data callbacks
        call vec_destroy(rhs)
        call mat_destroy(lhs)

      end do ! End of equation loop
      
    end do ! End of functional loop

    ! Now forget unnecessary variables
    ierr = adj_forget_adjoint_equation(adjointer, start_timestep)
    call adj_chkierr(ierr)

  end do ! End of timestep loop

  ! Forget everything
  ierr = adj_forget_adjoint_equation(adjointer, 0)
  call adj_chkierr(ierr)
\end{fortrancode}
\end{minipage}
\end{boxwithtitle}


\section{Replaying the forward run} \label{sec:replay}

\libadjoint can not only assemble the adjoint equations but also the forward equations. 
This can be used to replay the forward simulation, which provides a very powerful debugging tool for checking the correctness of the model annotation and the callback implementation, see \autoref{sec:compare}.

The function interface for assembling a forward equation is very similar to the one for the adjoint equation. 
Therefore, the adjoint main loop template given in \autoref{sec:typical_adjoint_main_loop} can be easily adopted to a forward main loop.

\defapis{adj_get_forward_equation}

\begin{boxwithtitle}{\texttt{adj_get_forward_equation}}
\begin{minipage}{\columnwidth}
\begin{ccode}
  int adj_get_forward_equation(adj_adjointer* adjointer, int equation, 
                               adj_matrix* lhs, adj_vector* rhs, 
                               adj_variable* variable);
\end{ccode}
\begin{fortrancode}   
  function adj_get_forward_equation(adjointer, equation, lhs, rhs, variable) 
           result(ierr) 
    type(adj_adjointer), intent(inout) :: adjointer
    integer(kind=c_int), intent(in), value :: equation
    type(adj_matrix), intent(out) :: lhs
    type(adj_vector), intent(out) :: rhs
    type(adj_variable), intent(out) :: variable
    integer(kind=c_int) :: ierr
  end function adj_get_forward_equation
\end{fortrancode}
\end{minipage}
\end{boxwithtitle}

This function assembles the forward equation with index \texttt{equation}.

For this to work, the source term callback has to be registered, see \refapi{adj_register_forward_source_callback}. 
The result is a linear system with operator \texttt{lhs} and right hand side \texttt{rhs}. 
Furthermore, the variable that this forward equation targets is provided as \texttt{fwd_var}.
The solution of the linear system must be recorded with \refapi{adj_record_variable}.
If the variable has already been recorded, for example if \refapi{adj_get_forward_equation} is used for a comparison against the original run, the storage settings have to been adjusted before the recording, see \autoref{sec:compare}.


\defapis{adj_evaluate_functional}

\begin{boxwithtitle}{\texttt{adj_evaluate_functional}}
\begin{minipage}{\columnwidth}
\begin{ccode}
  int adj_evaluate_functional(adj_adjointer* adjointer, int timestep, 
                              char* functional, adj_scalar* output)
\end{ccode}
\begin{fortrancode}   
  function adj_evaluate_functional(adjointer, timestep, functional, output) 
           result(ierr)
    type(adj_adjointer), intent(inout) :: adjointer
    integer, intent(in) :: timestep
    character(len=*), intent(in) :: functional
    adj_scalar_f, intent(out) :: output
    integer :: ierr
  end function adj_evaluate_functional
\end{fortrancode}
\end{minipage}
\end{boxwithtitle}

This function evaluates the functional named \texttt{functional} at timestep \texttt{timesteps} and saves the result to \texttt{output}.

It assumes that a functional callback with name \texttt{functional} is registered, see \refapi{adj_register_functional_callback}. 
If this callback is missing \refapi{ADJ_ERR_NEED_CALLBACK} is returned.
Furthermore, the evaluation of a functional without any dependencies returns \refapi{ADJ_WARN_UNINITIALISED_VALUE}. 
In both cases, \texttt{output} remains uninitialised.

